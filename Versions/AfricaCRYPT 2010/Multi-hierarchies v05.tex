\documentclass{llncs}

\usepackage{fullpage}
\usepackage{amsmath,amssymb}

%\include{../../commands}
\newcommand{\A}{\mathcal{A}}
\newcommand{\B}{\mathcal{B}}
\newcommand{\C}{\mathcal{C}}
\newcommand{\U}{\mathcal{U}}
\newcommand{\G}{\mathcal{G}}
\newcommand{\D}{\mathcal{D}}
\newcommand{\R}{\mathcal{R}}
\renewcommand{\S}{\mathcal{S}}

\newcommand{\Zbb}{\mathbb{Z}}
\newcommand{\Gbb}{\mathbb{G}}

\newcommand{\ID}{\mathit{ID}}
\newcommand{\TA}{\mathit{TA}}

\newcommand{\pk}{\mathit{pk}}
\newcommand{\sk}{\mathit{sk}}

\newcommand{\getsr}{\stackrel{{\scriptscriptstyle\$}}{\gets}}
\newcommand{\adv}[2]{\mathit{Adv}_{#1}^{\texttt{#2}}}

\newcommand{\param}{\mathit{param}}

\title{Broadcast Encryption with Multiple Trust Centers and Dynamic Coalitions}

\author{Kent D. Boklan\inst{1} \and
		Alexander W. Dent\inst{2} \and
		Christopher A. Seaman\inst{3}}

\institute{Queens College, City University of New York, USA \and
			Royal Holloway, University of London, UK \and
			Graduate Center, City University of New York, USA}

\begin{document}
\maketitle

\begin{abstract}
In this paper we extend the notion of hierarchical identity-based
encryption with wildcards (WIBE) from the domain of a single Trusted
Authority (TA) to a setting with multiple, independent Trusted
Authorities each with their own WIBE. In this multi-trust-authority
WIBE environment, a group of TA's may form coalitions, enabling
secure communication across domains. These coalitions may be
temporary or dynamic in the sense that individual coalition TA
members may be added or removed at any time. This allows the
broadcast of confidential messages to large groups of users within a
coalition with a single ciphertext. We provide a full syntax and
security model for multi-trust-authority WIBEs, and give a
constructions based on the Boneh-Boyen WIBE scheme for both passive
and active attackers.
\end{abstract}

\section{Introduction} \label{sec:intro}

Identity-based encryption (IBE) \cite{Shamir:84:Identity} is an
attractive alternative to public-key encryption. Public-key
encryption (PKE) allows for encryption only after obtaining a
certified copy of the recipient�s public key. In identity-based
encryption, a message is encrypted using a personal identifier such
as an email address rather than using a certified public key
obtained from a public-key infrastructure. The public key
infrastructure is replaced by an identification system and a single
public set of common parameters.

A hierarchical identity-based encryption (HIBE)
\cite{Horwitz:02:Towards} is an extension of identity-based
encryption with identities organised in a tree with a single root
master key. Thus, identities in a HIBE are a sequence of bitstrings,
e.g.\ (``City University of New York'', ``Computer Science'', ``Bob
Smith''). The identity hierarchy must mirror the trust hierarchy
within an organisation as the private key for any identity can be
used to generate a private key for any subordinate identity.
One-to-many encryption may be achieved within a hierarchy by using a
hierarchical identity-based encryption with wildcards (WIBE)
\cite{Abdalla:06:Identity}. In a WIBE scheme, encryption is
performed with respect to a pattern of identities and wildcards. An
identity is said to match the pattern if every non-wildcard level of
the pattern is the same as the identity. A ciphertext can be
decrypted by any user whose identity matches the pattern under which
the message was encrypted.

We present an extension of the WIBE concept to a situation with
multiple trusted authorities. Multiple-trust-authority WIBE
(MTA-WIBE) schemes allow trust authorities to implement instances of
the WIBE scheme (independent of any other trust authority). However,
an MTA-WIBE allows trust authorities to form coalitions which enable
secure cross-domain communications. These coalitions may be
temporary or dynamic in the sense that individual coalition members
may be added or removed at any time. The WIBE functionality allow
encrypted messages to be sent to a large set of users within the
coalition simultaneously. A ciphertext can only be decrypted by a
user if (a) their trust authority is in the coalition, and (b) their
user identity matches the pattern under which the ciphertext was
encrypted. Coalition formation is achieved by a two-stage process in
which trust authorities first exchange (public) messages and then
broadcast (public) ``update'' messages to members of their
hierarchy. This allows members to form \emph{coalition decryption
keys} which allow ciphertexts to be decrypted.

Other researchers have considered the question of developing IBE
systems with multiple trusted authorities. For example, a scheme
proposed by Paterson and Srinivasan \cite{Paterson:08:Security} and
another by Boklan \emph{et al.} \cite{Boklan:08:Flexible} introduced
multiple-trust-authority IBE systems which allow trusted authorities
to interact in order to derive a shared IBE system. The Paterson and
Srinivasan \cite{Paterson:08:Security} paper constructed an IBE
scheme which supported multiple trust authorities in a way which
makes it infeasible for an attacker to determine which trust
authority's public parameters was used to form the ciphertext -�
i.e.\ the ciphertext preserves the anonymity of the trust authority.
However, the Paterson and Srinivasan scheme does not allow trust
authorities to form trust coalitions. The Boklan \emph{et al.}
scheme \cite{Boklan:08:Flexible} allows trust authorities to
cooperate to form trust coalitions, in the sense that within the
coalition a private key issued by $\TA_i$ for an identity $\ID$ can
be translated into a private key issued by $\TA_j$ for the same
identity. However, in order to achieve this functionality, the
scheme requires that the coalition trust authorities setup their
master private keys simultaneously. Furthermore, every trust
authority can deduce the master private key of every other trust
authority. This is clearly a disadvantage in any setting where the
trust authorities share anything less than complete trust in each
other. Unlike our scheme, neither earlier scheme supports dynamic
coalitions, hierarchical identity-based encryption, or one-to-many
communication.

We provide a full syntax and security model for an MTA-WIBE scheme.
We give a selective-identity instantiation based on the Boneh-Boyen
WIBE scheme \cite{Abdalla:06:Identity,Boneh:04:Efficient}. We also
give generic methods to transform a selective-identity scheme into a
fully secure scheme, and to transform a passively secure scheme into
an actively secure scheme. This latter transform is a novel
implementation of the Boneh-Katz transform \cite{Boneh:05:Improved}
and is the most efficient generic transform for creating IND-CCA2
secure WIBEs or MTA-WIBEs from IND-CPA WIBEs or MTA-WIBEs in the
literature.

\medskip \noindent
\textbf{Usage Scenarios} We believe that MTA-WIBEs are useful in a
variety of contexts. For example, suppose that a number of
universities (e.g.\ ``NYU'', ``CUNY'', and ``ENS'') have independent
MTA-WIBEs with a common naming convention (e.g.\ (SCHOOL,
DEPARTMENT, INDIVIDUAL)). If these universities were to form a
coalition, then a message encrypted using pattern (*,``Computer
Science'', *) could be decrypted by any member of the computer
science department in any of the three universities in the
coalition. Alternatively, a message encrypted using the pattern
(``NYU'', *, ``Bob Smith'') could only be decrypted by an individual
named Bob Smith at NYU. We believe that MTA-WIBEs can be used for
efficient communication across organisations (with common naming
conventions) for shared projects.

Another usage scenario is within the realm of hardware distribution.
Suppose a number of companies produce sensors for use in an ad-hoc
network (including, e.g., ``IBM'') and suppose that these sensors
can be classified according to the function they perform (including,
e.g., ``climate sensor''). We assume that there is a common naming
structure for these sensors (e.g.\ (MANUFACTURER, SENSOR TYPE,
PROJECT)). If the manufacturers agree to form a coalition, then a
message encrypted using the coalition parameters and the pattern (*,
``climate sensor'', ``Project Intercept'') could be decrypted by any
climate sensor on Project Intercept. Alternatively, a message
encrypted using the pattern (``IBM'', ``climate sensor'', *) can
only be decrypted by IBM's climate sensors. This provides a method
to address all sensors within a project (for project information
distribution) or to address all sensors produced by an individual
manufacturer (e.g.\ for software patching or update).

\section{Multiple-trust-authority WIBEs} \label{sec:mta-wibe}

Throughout the paper we will use standard notation for algorithms
and assignment -- see Appendix~\ref{app:standard:algorithms}.

\subsection{Syntax} \label{sec:mta-wibe:syntax}

A Trusted Authority is the root of a domain of trust with
responsibilities over the namespace of its organization. In general
we will refer to a Trusted Authority $\TA$ as a hierarchy of
identities of the form $\ID = (\ID_0, \ID_1,
\ldots,\ID_k)$ with the same first identity ($\ID_0 = \TA$) and maximum
depth of $L$. Given a population of $\TA$'s $\U = \{\TA_1, \TA_2,
\ldots, \TA_n\}$ we define a coalition $\C =
\{\TA_a, \TA_b, \ldots, \TA_{\ell}\} \subseteq \U$. We also define a pattern to be
a vector of identities and wildcards, i.e. $P=(P_{1},\ldots,P_k)$
where $P_{i} \in \{0,1\}^{*} \cup \{*\}$ for $1\leq i\leq k$. We say
that an identity $\ID = (\ID_{0},\ldots,\ID_{k})$ matches a pattern
$P=(P_{0},\ldots,P_{k'})$, written $\ID \in_{*} P$, if $k \leq k'$
and $P_{i} \in \{ \ID_{i}, *\}$ for all $1\leq i\leq k$. A
multi-trust-authority WIBE (MTA-WIBE) consists of a number PPT
algorithms. The following algorithms may be used by a TA:

\begin{itemize}
\item $\texttt{Setup}$: This algorithm produces public parameters to
be used by all trust authorities. This is written $\param \getsr
\texttt{Setup}(1^{k})$. These public parameters are assumed to be an
implicit input to all other algorithms.

\item $\texttt{CreateTA}$: This algorithm creates the master
public/private keys for a trust authority with a particular name
(``$\TA_i$''). The algorithm takes as input the proposed name
for the trusted authority (``$\TA_i$'') and outputs a master
key-pair $(\pk_i, \sk_i)$, written $(\pk_i,\sk_i) \getsr
\texttt{CreateTA}($``$\TA_i$''$)$.

\item $\texttt{CoalitionBroadcast}$: Once a set of trust
authorities agree to setup a coalition between them, each trust
authority runs this algorithm to produce the information which
allows the other trust authorities in the coalition to produce
coalition keys for the members of their hierarchy. For some
coalition $\C \subseteq \U$ containing $\TA_i$, trust authority
$\TA_i$ uses its secret key and the public keys of participating
authorities to generate public parameters specific to each other
authority. This is written as $W_{i} \getsr
\texttt{CoalitionBroadcast}(\TA_i,sk_{i}, \C, PK)$, where
$PK=\{\pk_{j}:\TA_{j}\in\C\}$ is the set of master public keys
in the coalition and $W_{i}=\{w_{i,j} : \TA_{j} \in \C \setminus
\TA_{i}\}$ is the set of key update elements. Each $w_{i,j}$ is sent from
$\TA_i$ to $\TA_j$.

\item $\texttt{CoalitionUpdate}$: After every member $\TA_{i}$ of the
coalition has provided a message $w_{i,j}$ to $\TA_j$, trust
authority $\TA_j$ uses this algorithm to combine those messages
to allow creation of coalition-specific secret keys. It outputs
a message $v_j$ to be broadcast to every member of $\TA_j$'s
hierarchy (who then run the \texttt{CoalitionExtract}
algorithm). This is written $v_j \getsr
\texttt{CoalitionUpdate}(\TA_j,sk_j, \C, \hat{W}_{j})$
where $\hat{W}_{j} = \{w_{i,j} : \TA_i \in \C \setminus
\TA_j\}$ is the set of coalition parameters received by
$\TA_{j}$.
\end{itemize}

We now describe the algorithms required by the individual users.

\begin{itemize}
\item \texttt{Extract}: This algorithm is used by an individual
to generate private keys for their subordinates (entities on the
level below them in the hierarchical structure). The keys
generated are specific to the TA's WIBE, although they may later
be adjusted for use in a coalition environment. For entity $\ID
= (\ID_0, \ID_1, \ldots, \ID_k)$ extracting a private key for
subordinate $\ID^{\dagger} = (\ID_0, \ID_1, \ldots,
\ID_k, \ID')$ the algorithm outputs $d_{\ID^{\dagger}} \getsr
\texttt{Extract}(\ID,d_{\ID}, \ID')$.

\item \texttt{CoalitionExtract}: Users in a trust authority's
hierarchy may use this algorithm to adapt their TA-specific WIBE
private key for use within a coalition. To accomplish this their
TA, $\TA_i$, must provide an adjustment parameter $v_i$ to be
combined with the user's private key $d_{\ID}$. A user generates
its coalition key as $c_{\ID}\gets
\texttt{CoalitionExtract}(d_{\ID}, v_i)$.

\item \texttt{Encrypt}: This algorithm can be used by an individual
to encrypt a message $m$ to any individual whose identity
matches a pattern $P$ in the coalition $\C$. This is computed as
$C \getsr \texttt{Encrypt}(\C,\mathit{PK},P,m)$ where $PK =
\{pk_i : \TA_i \in \C\}$. We assume that $\C = \{ P_{0}\}$ whenever
$P_{0} \neq *$.

\item \texttt{Decrypt}: This algorithm can be used to decrypt a
ciphertext $C$ under a coalition key $c_{\ID}$ and outputs
either a message $m$ or the error symbol $\perp$. We write this
operation as $\texttt{Decrypt}(\ID,c_{\ID},C)$. If no coalition
is currently defined, then $c_{\ID} \gets d_{\ID}$.
\end{itemize}

It is, of course, possible to extend the MTA-WIBE syntax so that
coalition update values $v_{j}$ are produced after a protocol
interaction between trust authorities in the coalition, but we use
the simpler broadcast case as it is sufficient for our
instantiation. We require that the scheme is correct in the obvious
sense that decryption ``undoes'' encryption for correctly generated
trust authorities and coalitions.

\subsection{Security Model} \label{sec:mta-wibe:security model}

We provide a security model for an MTA-WIBE. We begin by defining a
selective-identity sID-IND-CPA model.This is a game played between a
PPT attacker $\A=(\A_{0},\A_{1})$ and a hypothetical challenger: (1)
the attacker runs $(\C^{*},P^{*},\omega)\getsr
\A_{0}(1^{k})$ where $\C^{*}$ is the list of TA identifiers in the
challenge coalition, $P^{*}$ is the challenge pattern, and $\omega$
is state information; (2) the challenger generates $\param \getsr
\texttt{Setup}(1^{k})$ and $(\pk_{i},\sk_{i})\getsr \texttt{CreateTA}(\TA_{i})$ for all
$\TA_{i} \in \C^{*}$; (3) the attacker outputs a bit $b'\getsr
\A_{1}(\param,\mathit{PK},\omega)$ where $\mathit{PK} \gets \{ \pk_{i}
\,:\, \TA_{i} \in \C^{*}\}$. During $\A_{1}$'s execution, it may
access the following oracles:
\begin{itemize}
\item $\texttt{CreateTA}(\TA)$: The oracle computes $(\pk,\sk)
\getsr \texttt{CreateTA}(\TA)$ and returns $\pk$. This oracle can
only be queried for values of $\TA$ that do not already have an
associated public key. This TA is labelled ``honest''. (All TA's
in $\C^{*}$ are also labelled ``honest''.)

\item $\texttt{SubmitTA}(\TA,\pk)$: This oracle associates the
identity $\TA$ with the public key $\pk$. It is used to model
rogue TAs. This oracle can only be queried for values of $\TA$
that do not already have an associated public key. This TA is
labelled ``corrupt''.

\item $\texttt{CoalitionBroadcast}(\TA,\C)$: This oracle
computes the coalition key update set $W \getsr \allowbreak
\texttt{CoalitionBroadcast}(\TA,\sk,\C,\mathit{PK})$ where $\TA$ is
``honest'', $\sk$ is the private key associated with $\TA$, $\C$
is a coalition containing $\TA$, and $\mathit{PK}$ is the set of
public keys for trust authorities in $\C$. The oracle returns
the set $W = \{ w_{j} \,:\, \TA_{j} \in \C \setminus \TA \}$.

\item $\texttt{CoalitionUpdate}(\TA, \C, \hat{W})$: The
oracle computes the adjustment parameter $v\getsr
\texttt{CoalitionUpdate}(\TA,\sk,\C,\hat{W})$ where $\TA$ is honest, $\sk$ is the
private key associated with $\TA$, $\C$ is a coalition
containing $\TA$, and $\hat{W} = \{ w_{j} \,:\,
\TA_{j} \in \C\}$ is the set of key update messages that purport to
be from $\TA_{j}$. The oracle returns the value $v$. Note that
we do not require that $\hat{W}$ corresponds to the elements
returned by $\texttt{CoalitionBroadcast}$.

\item $\texttt{Corrupt}(\ID)$: The oracle returns $d_{\ID}$
for the identity $\ID$.  Note that if $\ID = \TA$ then this
method returns the private key $\sk$ associated with the trust
authority $\TA$. This oracle can only be queried for situations
where $\TA$ is ``honest''. If $\ID=\TA$ then $\TA$ is labelled
``corrupt''.

\item $\texttt{Encrypt}(m_{0},m_{1})$: The oracle returns $C^{*}
\getsr \texttt{Encrypt}(\C^{*},\mathit{PK},P^{*},m_{b})$ where
$\mathit{PK}$ is the set of public keys associated with trust
authorities $\TA \in \C$. This oracle can only be queried once
and only on values with $|m_{0}|=|m_{1}|$.
\end{itemize}
The attacker is forbidden from corrupting an identity $\ID \in_{*}
P^{*}$ under a trust authority $\TA \in \C^{*}$. The attacker's
advantage is defined to be
\begin{displaymath}
\adv{\A}{sID}(k) = |\Pr[b'=1 \,|\, b=1] - \Pr[b'=1 \,|\, b=0]| \, .
\end{displaymath}

We define extended notions of security in the usual way. The IND-CPA
notion of security is identical to the sID-IND-CPA notion of
security except that there is no $\A_{0}$ algorithm. The algorithm
$\A_{1}$ takes $1^{k}$ as input and the encryption oracle changes so
that it works as follows:
\begin{itemize}
\item $\texttt{Encrypt}(\C^{*},P^{*},m_{0},m_{1})$: The oracle returns $C^{*}
\getsr \texttt{Encrypt}(\C^{*},\mathit{PK},P^{*},m_{b})$ where
$\mathit{PK}$ is the set of public keys associated with trust
authorities $\TA \in \C$. This oracle can only be queried once,
only on values with $|m_{0}|=|m_{1}|$, and only on coalitions
$\C$ where every $\TA \in \C$ is honest.
\end{itemize}
The IND-CCA2 notion of security is identical to the IND-CPA notion
of security except that $\A_{1}$ has access to a decryption oracle:
\begin{itemize}
\item $\texttt{Decrypt}(\C,\ID,C)$: This oracle checks whether the
coalition key $c_{\ID}$ has been defined for the coalition $\C$
(via a \texttt{CoalitionUpdate} oracle query). If not, the
oracle returns $\perp$. Otherwise, the oracle returns
$\texttt{Decrypt}(\ID,c_{\ID},C)$. This oracle can only be
queried on identities for which the trust authority
$\TA=\ID_{0}$ is honest.
\end{itemize}
The attacker is forbidden from submitting $(\C^{*},\ID,C^{*})$ to
the decryption oracle for any identity $\ID \in_{*} P^{*}$.

We note that this model allows for ``rogue TAs'' whose parameters
are generated maliciously, rather than by the \texttt{CreateTA}
oracle, as there is no requirement that $\C$ contain TA identities
generated by the \texttt{CreateTA} oracle \emph{except} for the
coalition submitted to the \texttt{Encrypt} oracle. The
\texttt{SubmitTA} oracle allows the attacker to define a public key
for a rogue TA although this public key is only used by the
\texttt{CoalitionBroadcast} oracle. We also note that the inability
to query an oracle to obtain the coalition key $c_{\ID}$ does not
represent a weakness in the model (assuming that $c_{\ID}$ and
$d_{\ID}$ are secured in a similar manner) as $c_{\ID}$ can always
be formed from $d_{\ID}$ and the public value $v$.

\section{Boneh-Boyen MTA-WIBE} \label{sec:BB construction}

We present a selective-identity IND-CPA secure MTA-WIBE based on the
Boneh-Boyen MTA-WIBE. The scheme is given in Figure~\ref{fig:BB
MTA-WIBE}. Our scheme makes use of two prime-order groups
$(\Gbb,\Gbb_{T})$ and an efficiently computable bilinear map $e :
\Gbb \times \Gbb \rightarrow \Gbb_{T}$. We assume that the size of
the prime $p$ is determined by the security parameter $k$.

\begin{figure}[!t]
\begin{multicols}{2}
\begin{minipage}{0.55\textwidth}
\begin{tabbing}
X\=X\=X\=\kill
$\texttt{Setup}(1^{k})$:\\
\> $g_{1}, g_{2}, u_{i,j} \getsr \Gbb^{*}$ for $0\leq i\leq L$, $j\in \{0,1\}$\\
\> $\param \gets (g_{1}, g_{2}, u_{0,0},\ldots,u_{L,1})$\\
\> Return $\param$\\
\\
$\texttt{CreateTA}(\TA)$:\\
\> $\alpha \getsr \Zbb_{p}$\\
\> $\pk \gets g_{1}^{\alpha}$; $\sk \gets g_{2}^{\alpha}$\\
\> Return $(\pk,\sk)$\\
\\
$\texttt{CoalitionBroadcast}(\TA,\sk,\C,\mathit{PK})$:\\
\> For each $\TA_{j} \in \C$:\\
\>\> $r_{j} \getsr \Zbb_{p}$\\
\>\> $w_{j,0} \gets g_{2}^{\alpha} (u_{0,0} \cdot u_{0,1}^{\TA_{j}})^{r_{j}}$\\
\>\> $w_{j,1} \gets g_{1}^{r_{j}}$\\
\>\> $w_{j} \gets (w_{j,0},w_{j,1})$\\
\> $W \gets \{ w_{j} \,:\, \TA_{j} \in \C\setminus\{\TA\}\}$\\
\> Return $W$\\
\\
$\texttt{CoalitionUpdate}(\TA,\sk,\C,\hat{W})$:\\
\> Parse $\hat{W}$ as $\{ w_{j} \,:\, \TA_{j} \in \C\}$\\
\> Parse $w_{j}$ as $(w_{j,0},w_{j,1})$\\
\> $v_{0} \gets \prod_{\TA_{j}\in \C\setminus\{\TA\}} w_{j,0}$\\
\> $v_{1} \gets \prod_{\TA_{j}\in \C\setminus\{\TA\}} w_{j,1}$\\
\> $v \gets (v_{0},v_{1})$\\
\> Return $v$\\
\\
$\texttt{CoalitionExtract}(d_{\ID},v)$:\\
\> Parse $v$ as $(v_{0},v_{1})$\\
\> Parse $d_{\ID}$ as $(h,a_{0},\ldots,a_{k})$\\
\> $h' \gets h \cdot v_{0}$\\
\> $a'_{0} \gets a_{0} \cdot v_{1}$\\
\> Return $c_{\ID} \gets (h',a'_{0},a_{1},\ldots,a_{k})$
\end{tabbing}
\end{minipage}
\vfill
\begin{minipage}{0.45\textwidth}
\begin{tabbing}
X\=X\=X\=\kill
$\texttt{Extract}(\ID,d_{\ID},\ID')$:\\
\> If $\ID = \varepsilon$ then\\
\>\> $r_{0}, r_{1} \getsr \Zbb_{p}$\\
\>\> $h \gets g_{2}^{\alpha} (u_{0,0} \cdot u_{0,1}^{\ID_{0}})^{r_{0}} (u_{1,0} \cdot u_{1,1}^{\ID_{1}})^{r_{1}}$\\
\>\> $a_{0} \gets g_{1}^{r_{0}}$; $a_{1} \gets g_{1}^{r_{1}}$\\
\>\> Return $(h,a_{0},a_{1})$\\
\> Else\\
\>\> Parse $d_{\ID}$ as $(h,a_{0},\ldots,a_{k})$\\
\>\> $r_{k+1} \getsr \Zbb_{p}$\\
\>\> $h' \gets h (u_{k+1,0} u_{k+1,1}^{\ID'})^{r_{k+1}}$\\
\>\> $a_{k+1} \gets g_{1}^{r_{k+1}}$\\
\>\> Return $(h',a_{0},\ldots,a_{k+1})$\\
\\
$\texttt{Encrypt}(\C,\mathit{PK},P,m)$:\\
\> $t\getsr \Zbb_{p}$\\
\> $C_{1} \gets g_{1}^{t}$\\
\> If $i\in W(P)$ then $C_{2,i} \gets (u_{i,0}^{t},u_{i,1}^{t})$\\
\> If $i\notin W(P)$ then $C_{2,i} \gets (u_{i,0} \cdot u_{i,1}^{P_{i}})^{t}$\\
\> $C_{3} \gets m \cdot e(\prod_{TA_{j} \in \C} \pk_{j} , g_{2})^{t}$\\
\> Return $(C_{1},C_{2,1},\ldots,C_{2,\ell},C_{3})$\\
\\
$\texttt{Decrypt}(\ID,c_{\ID},C)$:\\
\> Parse $c_{\ID}$ as $(h,a_{0},\ldots,a_{\ell})$\\
\> Parse $C$ as $(C_{1},C_{2,1},\ldots,C_{2,\ell},C_{3})$\\
\> If $i\in W(P)$ then\\
\>\> Parse $C_{2,i}$ as $(\tilde{u}_{0},\tilde{u}_{1})$\\
\>\> $C'_{2,i} \gets \tilde{u}_{0} \cdot \tilde{u}_{1}^{\ID_{i}}$\\
\> If $i\notin W(P)$ then $C'_{2,i} \gets C_{2,i}$\\
\> $m' \getsr C_{3} \cdot \prod_{i=1}^{\ell} e(a_{i},C'_{2,i}) / e(C_{1},h)$\\
\> Return $m'$
\end{tabbing}
\end{minipage}
\end{multicols}
\caption{The Boneh-Boyen MTA-WIBE. Recall that any identity $\ID$ has $\ID_{0} = \TA$. The \texttt{Extract} algorithm
differentiates between initial key extraction by the $\TA$ and
hierarchical extraction by a user. The \texttt{Decrypt} algorithm assumes that the depth
of the decryption key and the depth of the ciphertext are equal. If the depth of the
decryption key is shorter than the depth of the ciphertext, then the user can extract a key
of a correct length and use the decryption algorithm.}
\label{fig:BB MTA-WIBE}
\end{figure}

We prove this algorithm secure in the sID-IND-CPA security model.
The intuition behind the proof is that any coalition of trust
authorities can be viewed as an extended hierarchy with a ``ghost''
trust authority at the top level. Each actual trust authority is
represented as a first-level identity under this ghost TA and,
through communication with the other trust authorities in the
coalition, is able to determine a private key for their first-level
identity under the ghost TA. More specifically, if we consider a
coalition $\C = \{ \TA_{1},\ldots,\TA_{n}\}$ in which each TA has a
private key $\sk_{i} = g_{2}^{\alpha_{i}}$, then the ghost TA will
have a private key $g_{2}^{\sum \alpha_{i}}$. Upon forming the
coalition, the trust authority $\TA_{j}$ receives the messages
\begin{displaymath}
w_{i,j,0} \gets g_{2}^{\alpha_{i}} (u_{0,0} \cdot
u_{0,1}^{\TA_{j}})^{r_{i}} \qquad w_{i,j,1} \gets g_{1}^{r_{i}}
\end{displaymath}
from each $\TA_{i} \in \C \setminus \{TA_{j}\}$. This allows
$\TA_{j}$ to form the private key
\begin{displaymath}
h \gets g_{2}^{\sum_{i} \alpha_{i}} (u_{0,0} \cdot
u_{0,1}^{\TA_{j}})^{\sum_{i\neq j} r_{i}} \qquad a_{1} \gets
g_{1}^{\sum_{i\neq j} r_{i}}
\end{displaymath}
which is precisely the key that would be obtained if the ghost TA
were to distribute a private key to the identity $\TA_{j}$. The
security of the multi-TA scheme then essentially follows from the
security of the single-TA WIBE scheme, although care must be taken
to show that the broadcast messages $w_{i,j}$ and $v_{i}$ do not
leak information about the private keys to the attacker.

We prove this theorem in two steps. The first step removes the
wildcards to form an sID-IND-CPA secure MTA-HIBE (i.e.\ a WIBE
scheme which doesn't support wildcards).

\begin{theorem} \label{thm:BB-WIBE => BB-HIBE}
Suppose that there exists a PPT attacker $\A$ against the
sID-IND-CPA security of the multi-TA Boneh-Boyen WIBE  with
advantage $\adv{\A}{WIBE}(k)$. Then there exists a PPT attacker $\B$
against the sID-IND-CPA security of the multi-TA Boneh-Boyen HIBE
with advantage $\adv{\B}{HIBE}(k) = \adv{\A}{WIBE}(k)$.
\end{theorem}

Theorem~\ref{thm:BB-WIBE => BB-HIBE} is proven using the projection
technique of Abdalla \emph{et al.} \cite{Abdalla:06:Identity}. For
completeness a proof is given in Appendix~\ref{app:WIBE->HIBE}. The
more interesting step is to show that the HIBE is secure. This is
shown relative to the DBDH assumption:

\begin{definition}
Let $(\Gbb,\Gbb_{T})$ be groups of cyclic groups of prime order
$p(k)$ with a bilinear map $e$ and let $g$ be a generator of $\Gbb$.
Let $D_{k}$ be the distribution $\vec{x} \gets
(g,g^{a},g^{b},g^{c},e(g,g)^{abc})$ for $a,b,c \getsr \Zbb_{p}$. Let
$R_{k}$ be the distribution $\vec{x}\gets (g,g^{a},g^{b},g^{c},Z)$
for $a,b,c \getsr \Zbb_{p}$ and $Z\getsr \Gbb_{T}$. An algorithm
$\A$ has advantage
\begin{displaymath}
\adv{\A}{DBDH}(k) = |\Pr[\A(\vec{x})=1 \,|\, \vec{x} \getsr D_{k}] -
\Pr[\A(\vec{x})=1 \,|\, \vec{x} \getsr R_{k}]| \, .
\end{displaymath}
The DBDH assumption holds if every PPT attacker has negligible
advantage.
\end{definition}

\begin{theorem}
Suppose that there exists a PPT attacker $\A$ against the
sID-IND-CPA security of the Boneh-Boyen HIBE that makes at most
$q_{K}$ \emph{\texttt{Corrupt}} oracle queries and has advantage
$\adv{\A}{HIBE}(k)$. Then there exists a PPT algorithm $\B$ that
solves the DBDH problem with advantage $\adv{\B}{BDDH}(k)
\geq \adv{\A}{HIBE}(k)/2 -q_{K}/2p$.
\end{theorem}

\noindent \emph{Proof} We directly describe the algorithm $\B$ against the DBDH problem:
\begin{enumerate}
\item $\B$ receives the input $(g,g^{a},g^{b},g^{c},Z)$.
\item $\B$ runs $\A_{0}$ to obtain the challenge coalition
$\TA^{*}=\{\TA^{*}_{1},\ldots,\TA^{*}_{n^{*}}\}$ and the
challenge identity
$\ID^{*}=(\ID^{*}_{0},\ldots,\ID^{*}_{\ell^{*}})$ under the
challenge trust authority $\ID^{*}_{0} = \TA^{*}_{1}$ (wlog).
\item If $\ell^{*} < L$ then $\B$ randomly generates $\ID^{*}_{\ell^{*}+1},\ldots,\ID^{*}_{L}\getsr \Zbb_{p}$.
\item $\B$ computes the challenge parameters
    \begin{displaymath}
    \begin{array}{c}
    g_{1} \gets g \qquad g_{2}\gets g^{b} \qquad k_{i,j}, \alpha_{j} \getsr \Zbb_{p}^{*} \mbox{ for } 0\leq i\leq L, j\in \{0,1\}\\
    pk_1 \gets g^{a}/g^{\sum_{j=2}^{n^{*}} \alpha_{j}} \qquad pk_j \gets g^{\alpha_{j}}\mbox{ and } \sk_{j} \gets (g^{b})^{\alpha_{j}} \mbox{ for } 2\leq j\leq n^{*}\\
    u_{i,0} \gets g_{1}^{k_{i,0}}\cdot (g^{a})^{-\ID^{*}_{i}k_{i,1}} \qquad u_{i,1} \gets (g^{a})^{k_{i,1}} \qquad \mbox{ for } 0\leq i\leq L
    \end{array}
    \end{displaymath}
\item $\B$ runs $\A_{1}$ on the input $(g_{1},g_{2},u_{0,0},u_{0,1},\ldots,u_{L,0},u_{L,1},\mathit{PK})$ where
    $\mathit{PK}=(\pk_{1},\ldots,\pk_{n^{*}})$. If $\A_{1}$
    makes an oracle query, then $\B$ answers queries as follows:
    \begin{itemize}
    \item $\texttt{CreateTA}$: $\B$ generates $\alpha_{\TA} \getsr \Zbb_{p}$ and returns the public
    key $g_{1}^{\alpha_{\TA}}$, while storing $\alpha_{\TA}$
    for future use. Note that $\B$ knows the private key for
    all TAs except $\TA^{*}_{1}$. Hence, we only have to
    show how to simulate the remaining oracles for
    $\TA_{1}$.

    \item $\texttt{SubmitTA}$: $\B$ ignores any queried made to this
    oracle (as the Boneh-Boyen scheme does not make use of
    public key values in the $\texttt{CoalitionBroadcast}$
    algorithm).

    \item $\texttt{Corrupt}$: Suppose $\A$ requests the decryption key
    for $(\TA^{*}_{1},\ID_{1}\ldots,\ID_{\ell})$. If $\ID$
    is not ancestor of
    $(\TA^{*}_{1},\ID^{*}_{1},\ldots,\ID^{*}_{L})$, then
    there exists an index $1\leq \mu \leq
    \ell$ such that $\ID_{\mu} \neq \ID^{*}_{\mu}$. $\B$
    generates $r_{1},\ldots,r_{\mu}\getsr \Zbb_{p}$ and
    computes the decryption key $(h,a_{0},\ldots,a_{\mu})$
    for $(\ID_{0},\ldots,\ID_{\mu})$ as
        \begin{eqnarray*}
        h &\gets& g_{2}^{-\frac{k_{\mu,0}}{k_{\mu,1}(\ID_{\mu}-\ID^{*}_{\mu})}} \cdot g_{2}^{-\sum_{j=2}^{n^{*}}\alpha_{j}}\cdot \prod_{i=0}^{\mu} \left(u_{i,0}\cdot u_{i,1}^{\ID_{i}}\right)^{r_{i}} \\
        a_{i} &\gets& g_{1}^{r_{i}} \mbox{ for } 0\leq i\leq \mu-1 \\
        a_{\mu} &\gets& g_{2}^{-\frac{1}{k_{\mu,1}(\ID_{\mu}-\ID^{*}_{\mu})}} \cdot g_{1}^{r_{j}} \, .
        \end{eqnarray*}
        $\B$ computes the decryption key for $\ID$ using the
        key derivation algorithm and returns the result. If
        no such $\mu$ exists (i.e.\ if $\ID$ is an ancestor
        of $(\TA^{*}_{1},\ID^{*}_{1},\ldots,\ID^{*}_{L})$
        then $\B$ aborts .

    \item $\texttt{CoalitionBroadcast}$: Suppose that $\A$ makes
    a request for $\TA^{*}_{1}$ to send messages to the
    coalition $\C$. $\B$ generates $r_{1}\getsr
    \Zbb_{p}$ and computes for each $\TA_{i} \in \C \setminus
    \{\TA\}$
        \begin{eqnarray*}
        w_{i,0} &\gets& g_{2}^{-\frac{k_{1,0}}{k_{1,1}(\TA_{i}-\TA^{*}_{1})}} \cdot g_{2}^{-\sum_{j=2}^{n^{*}}\alpha_{j}} \cdot \left( u_{1,0} \cdot u_{1,1}^{\TA_{i}} \right)^{r_{1}}\\
        w_{i,1} &\gets& g_{2}^{-\frac{1}{k_{1,1}(\TA_{i}-\TA^{*}_{1})}}g_{1}^{r_{1}}
        \end{eqnarray*}
        and sets $w_{j}\gets (w_{j,0},w_{j,1})$. $\B$
        returns the list $\{ w_{j} : \TA_{j} \in \C
        \setminus \{\TA\}\}$.

    \item $\texttt{CoalitionUpdate}$: The output of this oracle
    can be returned directly as it is independent of any
    private key values.
    \item $\texttt{Encrypt}$: Suppose $\A_{1}$ makes the oracle
    query on two equal-length messages $(m_{0},m_{1})$. $\B$
    chooses a bit $b\getsr \{0,1\}$ and computes the
    ciphertext
        \begin{displaymath}
        C^{*} \gets (g^{c},(g^{c})^{k_{1,0}}, \ldots,
        (g^{c})^{k_{\ell^{*},0}}, m_{b}\cdot Z) \, .
        \end{displaymath}
    \end{itemize}
    $\A_{1}$ terminates with the output of a bit $b'$.
\item If $b=b'$ then $\B$ outputs 1. Otherwise, outputs 0.
\end{enumerate}

The $\texttt{Corrupt}$ oracle for subordinates works perfectly
provided that $\A_{1}$ does not abort, which can only occur if
$\A_{1}$ makes query on an identity $\ID$ which is not an ancestor
of $\ID^{*}$ but is an ancestor of $(\ID^{*}_{0}, \ldots,
\ID^{*}_{L})$. This can only occur if $\ell > \ell^{*}$ and $\ID_{\ell^{*}+1} =
\ID^{*}_{\ell^{*}+1}$, which occurs with probability $1/p$ as this
value is information theoretically hidden from $\A$. Hence, the
probability that this does not occur in the entire execution of $\A$
is $q_{K}/p$ where $q_{K}$ is the number of queries to the
$\texttt{Corrupt}$ oracle. We note that the corrupt oracle gives
correct responses for queries since
\begin{displaymath}
\sk_1 \left(u_{j,0}\cdot u_{j,1}^{\ID_{j}}\right)^{r} = g_{2}^{\frac{-k_{j,0}}{k_{j,1}(\ID_{j}-\ID^{*}_{j})}} \cdot g_{2}^{-\sum_{i=2}^{n^{*}} \alpha_{i}} \qquad \mbox{ for } \qquad r = -\frac{b}{k_{j,1}(\ID_{j}-\ID^{*}_{j})}\, .
\end{displaymath}
A similar calculation shows that the $\texttt{CoalitionBroadcast}$
algorithm gives correct broadcast messages for $\TA^{*}_{1}$. All
other oracles that $\B$ provides correctly simulate the security
model for $\A$.

If $Z=e(g,g)^{abc}$ then the challenge ciphertext is a correct
encryption of $m_{b}$. This is because an encryption using the
random value $c$ would have
\begin{eqnarray*}
& C_{1} = g_{1}^{c}=g^{c} & \\
& C_{2,i} = (u_{i,0} \cdot u_{i,1}^{\ID^{*}_{i}})^{c} = (g^{c})^{k_{i,0}} \qquad \mbox{ for } 0\leq i\leq \ell^{*} & \\
& C_{3} = m_{b} \cdot e(\prod_{i=0}^{n^{*}} \pk_i,g_{2})^{c}
= m_{b} \cdot e(g^{a},g^{b})^{c} = m_{b} \cdot e(g,g)^{abc} &
\end{eqnarray*}
The probability that $\B$ outputs 1 in this situation is the
probability that $b=b'$ in the sID-IND-CPA game for the attacker
$\A$. This probability can be shown to be $(\adv{\A}{HIBE}(k)-1)/2$.
If $Z$ is random then the challenge ciphertext information
theoretically hides $b$ and so the probability that $\B$ outputs 1
in this situation is $1/2$. This completes the proof. \qed

\section{Strengthened Security Results} \label{sec:CCA security}

\subsection{IND-CPA Security} \label{sec:CCA security:full CPA}

We may prove the security of the Boneh-Boyen scheme in the
(non-selective-identity) IND-CPA model by hashing all identities
before use. The proof of this fact mirrors the proof in Abdalla
\emph{et al.} \cite{Abdalla:06:Identity} (in the random oracle
model).

\subsection{IND-CCA2 Security} \label{sec:CCA security:CCA}

We may transform an IND-CPA MTA-WIBE scheme into an IND-CCA2
MTA-WIBE scheme using the CHK transform \cite{Canetti:04:Chosen} in
a manner similar to that described in Abdalla \emph{et al.}
\cite{Abdalla:06:Identity}. However, we will describe an alternative
transformation based on the Boneh-Katz (BK) transform
\cite{Boneh:05:Improved}. This gives a new and more efficient
method to transform IND-CPA secure WIBEs into IND-CCA2 secure WIBEs.

Boneh-Katz transforms an MTA-WIBE $\Pi$ into a new MTA-WIBE $\Pi'$
using a MAC algorithm $\textsc{Mac}$ and an ``encapsulation'' scheme
$(\G,\S,\R)$. The encapsulation scheme has a key generation
algorithm $\sigma \getsr \G(1^{k})$, commitment algorithm
$(K,\mathit{com},\mathit{dec}) \getsr \S(1^{k},\sigma)$, and a
decommitment algorithm $\D(\sigma,\mathit{com},\mathit{dec})$ which
outputs either a bitstring $K$ or the error symbol $\perp$. We
assume that $K \in \{0,1\}^{k}$. We require that if
$(K,\mathit{com},\mathit{dec}) \getsr \S(1^{k},\sigma)$ then
$\D(\sigma,\mathit{com},\mathit{dec})=K$. We also require that the
scheme is hiding in the sense that for all PPT attackers $\A$ have
negligible advantage:
\begin{displaymath}
%\adv{\A}{hide}(k) =
\Big| \Pr\Big[
\A(1^{k},\sigma,\mathit{com},K_{b}) = b \,:\,
\begin{array}{c}
\sigma \getsr \G(1^{k}); \, K_{0} \getsr \{0,1\}^{k}\\
(K_{1},\mathit{com},\mathit{dec})\getsr \S(1^{k},\sigma); \, b\getsr \{0,1\}
\end{array}
\Big]
- \frac{1}{2} \Big|
\end{displaymath}
We further require that the encapsulation scheme is binding in the
sense that for all PPT attackers $\A$ have negligible advantage:
\begin{displaymath}
%\adv{\A}{bind}(k) =
\Pr\Big[
\R(\sigma, \mathit{com}, \mathit{dec}') \notin \{ \perp, K\} \,:\,
\begin{array}{c}
\sigma \getsr \G(1^{k}); \, (K,\mathit{com},\mathit{dec})\getsr \S(1^{k},\sigma)\\
\mathit{dec}' \getsr \A(1^{k},\sigma,K,\mathit{com},\mathit{dec})
\end{array}
\Big]
\end{displaymath}
Lastly, we assume that the decommitments $\mathit{dec^{*}}$ are
always of some fixed size (which may depend on $k$). The security
notions for a MAC scheme are given in
Appendix~\ref{app:standard:MAC}. The transform of $\Pi$ into $\Pi'$
is given in Figure~\ref{fig:BK}. We assume that ``$-$'' represents
some fixed, publicly-known allowable identity for the CPA scheme; we
will deliberately exclude ``$-$'' from the space of allowable
identities in the CCA scheme.

\begin{figure}[!t]
\begin{multicols}{2}
\begin{minipage}{0.45\textwidth}
\begin{tabbing}
X\=X\=X\=\kill
$\texttt{Setup}'(1^{k})$:\\
\> $\param \getsr \texttt{Setup}(1^{k})$\\
\> $\sigma \getsr \G(1^{k})$\\
\> $\param' \gets (\param,\sigma)$\\
\> Return $\param'$\\
\\
$\texttt{Encode}(P,\alpha)$:\\
\> Parse $P$ as $(P_{0},\ldots,P_{\ell})$\\
\> For $i=0,\ldots,\ell$, $P'_{i} \gets P_{i}$\\
\> For $i=\ell+1,\ldots,L$, $P'_{i} \gets$ ``$-$''\\
\> $P'_{L+1} \gets \alpha$\\
\> Return $P'$\\
\\
$\texttt{Encode}'(P,\ID,\alpha)$:\\
\> For $i=1,\ldots,|P|-|\ID|$\\
\>\> If $P_{|\ID|+i} \neq *$ then $\ID'_{i} \gets P_{|\ID|+i}$\\
\>\> If $P_{|\ID|+i} = *$ then $\ID'_{i} \gets 1^{k}$\\
\> For $i=1,\ldots,L-|P|$\\
\>\> $\ID_{|P|-|\ID|+i} \gets$ ``$-$''\\
\> $\ID_{L-|\ID|+1} \gets \alpha$\\
\> Return $\ID'$
\end{tabbing}
\end{minipage}
\vfill
\begin{minipage}{0.45\textwidth}
\begin{tabbing}
X\=X\=X\=\kill
$\texttt{CoalitionExtract}'(d_{\ID},v)$:\\
\> $c_{\ID} \gets (d_{\ID},v)$\\
\> Return $c_{\ID}$\\
\\
$\texttt{Encrypt}'(\C,\mathit{PK},P,m)$:\\
\> $(K,\mathit{com},\mathit{dec}) \getsr \S(1^{k},\sigma)$\\
\> $P' \gets \texttt{Encode}(P,\mathit{com})$\\
\> $m' \gets (m,\mathit{dec})$\\
\> $C' \getsr \texttt{Encrypt}(\C,\mathit{PK},P',m')$\\
\> $\tau \gets \textsc{Mac}_{K}(\C\|P\|C')$\\
\> Return $(\mathit{com},\C,P,C',\tau)$\\
\\
$\texttt{Decrypt}'(c_{\ID},C)$:\\
\> Parse $C$ as $(\mathit{com},\C,P,C',\tau)$\\
\> Parse $c_{\ID}$ as $(d_{\ID},v)$\\
\> $\ID' \gets \texttt{Encode}'(P,\ID,\mathit{com})$\\
\> $d' \getsr \texttt{Extract}(\ID,d_{\ID},\ID')$\\
\> $c' \gets \texttt{CoalitionUpdate}(d',v)$\\
\> $(m,\mathit{dec}) \getsr \texttt{Decrypt}(c,C')$\\
\> $K \gets \R(\sigma,\mathit{com},\mathit{dec})$\\
\> If $\textsc{Mac}_{K}(\C\|P\|C')\neq \tau$ then return $\perp$\\
\> Else return $m$
\end{tabbing}
\end{minipage}
\end{multicols}
\caption{The Boneh-Katz transform for a MTA-WIBE. Any algorithm of $\Pi'$ not
explicitly defined in this figure is identical to the corresponding
algorithm in $\Pi$. The $\texttt{Encode}$ algorithm turns an $\ell$ level pattern
into an $L+1$ level pattern. The $\texttt{Encode}'$ algorithm computes
the extension to $\ID$ required to turn an identity which matches $P$ into
an identity which matches $\texttt{Encode}(P,\alpha)$.} \label{fig:BK}
\end{figure}

\begin{theorem} \label{thm:BK}
Suppose that the $\Pi$ is an IND-CPA secure MTA-WIBE, $\textsc{Mac}$
is an unforgeable MAC scheme, and $(\G,\S,\R)$ is a hiding and
binding encapsulation algorithm. Then the MTA-WIBE $\Pi'$ produced
by the BK transform is IND-CCA2 secure.
\end{theorem}

The proof strategy is similar to that of Boneh and Katz
\cite{Boneh:05:Improved} but has to deal with technical details
introduced by the trust authorities and the WIBE scheme. A full
proof is given in Appendix~\ref{app:BK}.

\subsection{Optimized BK for the Boneh-Boyen MTA-WIBE}

In the particular case where we apply the BK transform to the BB
MTA-WIBE, we can optimize the transform to obtain an efficient
sID-IND-CCA2 secure MTA-WIBE. The optimization comes from several
points:
\begin{itemize}
\item It is possible to extract a coalition key for an identity $\ID \| \ID'$
from $c_{\ID}$ and $\ID'$ using the same technique as for the
derivation of decryption key $d_{\ID}$. Hence we can run the
$\texttt{CoalitionExtract}$ algorithm as part of
$\texttt{CoalitionExtract}'$ algorithm, rather than as part of
the $\texttt{Decrypt}'$ oracle. This improves efficiency.
\item The BB MTA-WIBE is level independent in the sense that we can
specify an identifier for level $L$ (and extract a decryption
key which only works for that identifier at that level) without
needing to specify an identifier for the levels $1\leq i < L$.
Hence, we can specify a ciphertext with $C_{2,L}
\gets (u_{L,0}\cdot u_{L,1}^{\mathit{com}^{*}})^{t}$ without needing
to specify ciphertexts $C_{2,i} \gets (u_{i,0}\cdot
u_{i,1}^{\mbox{``}-\mbox{''}})^{t}$ for $|P| < i < L$. This
removes the need for the $\texttt{Encode}$ and
$\texttt{Encode}'$ algorithms, improves efficiency, and reduces
bandwidth.
%\item The BB MTA-WIBE has ciphertext which are extractable in the
%sense that given a ciphertext $C$ encrypted under pattern $P$ it
%is possible to extract a ciphertext $C'$ which is encrypted
%under pattern $P\|P'$ for any pattern $P'$. This means that we
%do not need to include the $(L+1)$-th level ciphertext component
%(where the identifier is specified as $\mathit{com}^{*}$) as
%this can easily be derived from the shorter ciphertext and
%$\mathit{com}^{*}$.
\end{itemize}

The (random-oracle-based) selective-identity to
non-selective-identity transform can then be applied to the
sID-IND-CCA2 scheme to give a IND-CCA2 secure scheme.

%\subsection*{Acknowledgements}
%
%This research was sponsored by the US Army Research Laboratory and
%the UK Ministry of Defence and was accomplished under Agreement
%Number W911NF-06-3-0001. The views and conclusions contained in this
%document are those of the authors and should not be interpreted as
%representing the official policies, either expressed or implied, of
%the US Army Research Laboratory, the US Government, the UK Ministry
%of Defence, or the UK Government. The US and UK Governments are
%authorized to reproduce and distribute reprints for Government
%purposes notwithstanding any copyright notation hereon. The research
%was conducted while the second author was visiting The Graduate
%Center of the City University of New York.

\bibliographystyle{plain}
\bibliography{Multi-hierarchies v05}

\appendix

\section{Standard Definitions} \label{app:standard}

\subsection{Algorithms and Assignment}
\label{app:standard:algorithms}

Throughout this article, $y\gets x$ denotes the assignment of the
value $x$ to the variable $y$ and $y\getsr S$ denotes the assignment
of a uniform random element of the finite set $S$ to the variable
$y$. If $\A$ is a probabilistic algorithm, then $y\getsr
\A(x)$ denotes the assignment of the output of the algorithm $\A$
run on the input $x$ to the variable $y$ when $\A$ is computed using
a fresh set of random coins. We write $y\gets \A(x)$ if $\A$ is
deterministic.

\subsection{MAC algorithms} \label{app:standard:MAC}

A MAC algorithm is a deterministic polynomial-time algorithm
$\textsc{Mac}$. It takes as input a message $m\in \{0,1\}^{*}$ and a
symmetric key $K\in \{0,1\}^{k}$, and outputs a tag $\tau \gets
\textsc{Mac}_{K}(m)$. It should be infeasible for a PPT attacker $\A$ to
win the unforgeability game: (1) the challenger generates a key
$K\getsr \{0,1\}^{*}$; (2) the attacker outputs a forgery
$(m^{*},\tau^{*}) \getsr \A(1^{k})$. During its execution the
attacker can query a MAC oracle with a message $m$ and will receive
$\textsc{Mac}_{K}(m)$. The attacker wins if
$\textsc{Mac}_{K}(m^{*})= \tau^{*}$ and $m^{*}$ was never queried to
the MAC oracle. The attackers probability of winning is written
$\adv{\A}{MAC}(k)$.

\section{Security Proof for BB-WIBE to BB-HIBE Reduction}
\label{app:WIBE->HIBE}

\noindent \emph{Proof of Theorem~\ref{thm:BB-WIBE => BB-HIBE}}

We directly describe the algorithm $\B$ which breaks the HIBE using
the algorithm $\A$ as a subroutine. Before we begin, we define some
useful notation. If $P=(P_{1},\ldots,P_{k})$ is a pattern, then
\begin{displaymath}
W(P) = \{ 1\leq i\leq k : P_{i} = *\} \quad \mbox{and}\quad
W(P_{\leq j}) =
\{ 1\leq i\leq \min\{j,k\} : P_{i} = *\} \, .
\end{displaymath}
The algorithm $\B$ runs as follows:
\begin{enumerate}
\item $\B$ runs $\A_{0}$ on the security parameter. $\A_{0}$ responds by
outputting a description of the challenge coalition $\TA^{*} =
(\TA^{*}_{1},\ldots,\TA^{*}_{n})$ and the challenge pattern
$P^{*} = (P^{*}_{1},\ldots,P^{*}_{\ell^{*}})$. Let $\pi$ be a
map which identifies the number of non-wildcard entries in the
first $i$ layers of $P^{*}$, i.e. $\pi(i) = i - |W(P^{*}_{\leq
i})|$. $\B$ outputs the challenge coalition $\TA^{*}$ and the
challenge identity
$\hat{\ID}^{*}=(\hat{\ID}^{*}_{1},\ldots,\hat{\ID}^{*}_{\pi(\ell^{*})})$
where $\hat{\ID}^{*}_{\pi(i)} = P^{*}_{i}$ for $i\notin
W(P^{*})$.
\item $\B$ responds with HIBE parameters $\mathit{param} = (\hat{g}_{1},\hat{g}_{2},\hat{u}_{1,0},\ldots,\hat{u}_{L,1})$. $\B$ generates WIBE parameters as follows:
    \begin{displaymath}
    \begin{array}{rcll}
    (g_{1},g_{2}) &\gets& (\hat{g}_{1},\hat{g}_{2})&\\
    u_{i,j} &\gets& \hat{u}_{\pi(i),j} &\mbox{ for } i\notin W(P^{*}), j\in\{0,1\}\\
    u_{i,j} &\gets& g_{1}^{\beta_{i,j}} &\mbox{ for } i\in W(P^{*}), j\in\{0,1\}\mbox{ where } \beta_{i,j}\getsr \Zbb_{p}\\
    u_{i,j} &\gets& \hat{u}_{i-|W(P^{*})|,j} &\mbox{ for } i\in \{\ell^{*}+1,\ldots,L\}, j\in\{0,1\} \, .
    \end{array}
    \end{displaymath}
\item $\B$ executes $\A_{1}$ on the public parameters $(g_{1},g_{2},u_{0,0},\ldots,u_{L,1})$. $\A$ may make the following oracle queries:
    \begin{itemize}
    \item \texttt{CreateTA}: $\B$ forwards this request to its own oracle and returns the response.
    \item \texttt{SubmitTA}: $\B$ may ignore these queries as the
    \texttt{CoalitionBroadcast} algorithm does not depend upon individual
    TA's public keys.
    \item \texttt{CoalitionBroadcast}: $\B$ forwards this request to its own oracle and returns the response.
    \item \texttt{CoalitionKeys}: $\B$ forwards this request to its own oracle and returns the response.
    \item \texttt{Corrupt}: To  extract a decryption key for an identity $\ID=(\ID_{1},\ldots,\ID_{\ell})$ which does not match the challenge pattern, $\B$ computes the projection of the identity onto the HIBE identity space to give a projected identity $\hat{\ID}=(\hat{\ID}_{1}\ldots, \hat{\ID}_{\hat{\ell}})$.
        \begin{itemize}
        \item If $\ell \leq \ell^{*}$ then $\hat{\ell} \gets \pi(\ell)$ and $\hat{\ID}_{\pi(i)} \gets \ID_{i}$ for $i\notin W(P^{*}_{\leq \ell})$. Since $\ID$ does not match the challenge pattern for the WIBE, $\hat{\ID}$ does not match the challenge identity for the HIBE. $\B$ queries its $\texttt{Corrupt}$ oracle on $\hat{\ID}$ and receives $(\hat{h},\hat{a}_{0},\ldots,\hat{a}_{\hat{\ell}})$ in response. $\B$ now ``retro-fits'' to find a complete key, by setting
            \begin{displaymath}
            \begin{array}{rcl}
            a_{0} &\gets& \hat{a}_{0}\\
            a_{i} &\gets& \hat{a}_{\pi(i)} \qquad \;\mbox{ for }1\leq i\leq \ell \mbox{ and } i\notin W(P^{*}_{\leq \ell})\\
            a_{i} &\gets& g_{1}^{r_{i}} \qquad \quad \mbox{ for }1\leq i\leq \ell \mbox{ and } i\in W(P^{*}_{\leq \ell})\mbox{ where } r_{i}\getsr \Zbb_{p}\\
            h &\gets& \hat{h} \prod_{i=1, i\in W(P^{*}_{\leq \ell})}^{\ell} (u_{i,0} \cdot u_{i,1}^{\ID_{i}})^{r_{i}}
            \end{array}
            \end{displaymath}
            and returning the key
            $(h,a_{0},\ldots,a_{\ell})$.
        \item If $\ell > \ell^{*}$, then $\hat{\ell} = \ell - |W(P^{*})|$, $\hat{\ID}_{\pi(i)} \gets \ID_{i}$ for $1\leq i\leq \ell^{*}$ and $i\notin W(P^{*})$, and $\hat{\ID}_{i-|W(P^{*})|} \gets \ID_{i}$ for $\ell^{*} < i\leq \ell$. Since $\ID$ does not match the challenge pattern for the WIBE, $\hat{\ID}$ does not match the challenge identity for the HIBE. $\B$ queries its $\texttt{Corrupt}$ oracle on $\hat{\ID}$ and receives $(\hat{h},\hat{a}_{0},\ldots,\hat{a}_{\hat{\ell}})$ in response. $\B$ now ``retro-fits'' to find a complete key, by setting
            \begin{displaymath}
            \begin{array}{rcl}
            a_{0} &\gets& \hat{a}_{0}\\
            a_{i} &\gets& \hat{a}_{\pi(i)} \qquad \qquad \; \,\mbox{ for }1\leq i\leq \ell^{*} \mbox{ and } i\notin W(P^{*})\\
            a_{i} &\gets& g_{1}^{r_{i}} \qquad \qquad \quad \, \mbox{ for }1\leq i\leq \ell^{*} \mbox{ and } i\in W(P^{*})\mbox{ where } r_{i}\getsr \Zbb_{p}\\
            a_{i} &\gets& \hat{a}_{i-|W(P^{*})|} \qquad \mbox{ for } \ell^{*} <i\leq \ell\\
            h &\gets& \hat{h} \prod_{i=1, i\in W(P^{*})}^{\ell^{*}} (u_{i,0} \cdot u_{i,1}^{\ID_{i}})^{r_{i}}
            \end{array}
            \end{displaymath}
            and returning the key
            $(h,a_{0},\ldots,a_{\ell})$.
        \end{itemize}
        \item \texttt{Encrypt}: $\A$ outputs two equal-length messages $(m_{0},m_{1})$. $\B$ queries its own encryption oracle
        on the messages $(m_{0},m_{1})$ and receives the
        ciphertext
        $(C^{*}_{1},\hat{C}^{*}_{2,1},\ldots,\hat{C}^{*}_{2,\pi(\ell^{*})},C^{*}_{3})$.
        $\B$ retro-fits this to form the challenge
        ciphertext for $\A$ by setting
                \begin{displaymath}
                \begin{array}{rcll}
                C^{*}_{2,i} &\gets& \hat{C}^{*}_{2,\pi(i)} & \qquad \mbox{ for } 1\leq i\leq \ell^{*}, i\notin W(P^{*})\\
                C^{*}_{2,i} &\gets& ({C^{*}_{1}}^{\beta_{i,0}},{C^{*}_{1}}^{\beta_{i,1}}) & \qquad \mbox{ for } 1\leq i\leq \ell^{*}, i\in W(P^{*}) \, .
                \end{array}
                \end{displaymath}
        \end{itemize}
    $\A_{1}$ terminates with the output of a bit $b'$.
    \item $\B$ outputs the bit $b'$.
\end{enumerate}
The algorithm $\B$ correctly simulates the oracles to which $\A$ has
access; furthermore, $\B$ wins the HIBE game if and only if $\A$
wins the game. Hence, the theorem is proven. \qed

\section{Security Proof for the BK Transform} \label{app:BK}

\noindent \emph{Proof of Theorem~\ref{thm:BK}}

Our proof proceeds through a series of games. Let $W_{i}$ be the
event that the attacker $\A$ outputs $b'=b$ in Game $i$ and let
starred values denote values computed during the computation of the
challenge ciphertext. Let Game 1 be the normal IND-CCA2 game for
$\A$. Hence,
\begin{displaymath}
\adv{\A}{CCA}(k) = 2 \cdot |\Pr[W_{1}] - 1/2| \, .
\end{displaymath}
Let Game 2 be the same as Game 1 except that $\A$ is deemed to lose
if it submits a ciphertext $(\mathit{com}^{*},P,C,\tau)$ such that
the decommitment value $\mathit{dec}'$ recovered during the
decryption process satisfies $\R(\sigma,\mathit{com},\mathit{dec}')
\notin \{ \perp, K^{*}\}$. It is easy to show that there exists a
PPT algorithm $\B$ such that $|\Pr[W_{1}] - \Pr[W_{2}]| \leq
\adv{\B}{bind}(k)$.

%Let Game 3 be identical to Game 2 except that the attacker is deemed
%to lose if it submits a ciphertext $(\mathit{com}^{*},P,C',\tau')$
%to the decryption oracle before querying the encryption oracle.
%Since $C'$ is an encryption under $\Pi$ we have that
%$|\mathit{Adv}_{3}-\mathit{Adv}_{2}|\leq \gamma(k)$.

Let Game 3 be identical to Game 2 except that the encryption oracle
computes $m'\phantom{}^{*} \gets (m_{b},0^{|\mathit{dec^{*}}|})$
rather than $m'\phantom{}^{*} \gets (m_{b},\mathit{dec^{*}})$. Let
$E_{2}$ be the event that $\A$ submits a ciphertext
$(\mathit{com}^{*},\C,P,C',\tau)$ to the decryption oracle with
$\textsc{Mac}_{K^{*}}(\C\|P\|C')=\tau$ in Game 2. Let $E_{3}$ be the
event that $\A$ submits a ciphertext
$(\mathit{com}^{*},\C,P,C',\tau)$ to the decryption oracle with
$\textsc{Mac}_{K^{*}}(\C\|P\|C')=\tau$ in Game 3. There exists an
algorithm $\B^{*}$ against the IND-CPA security of $\Pi$ such that
$|\Pr[E_{3}]-\Pr[E_{2}]| \leq
\adv{\B^{*}}{CPA}(k)$. The attacker $\B^{*}(\param)$ is defined as follows:
\begin{enumerate}
\item $\sigma \getsr \G(1^{k})$ and $(K^{*},\mathit{com}^{*},\mathit{dec}^{*})\getsr
\S(1^{k},\sigma)$.
\item $\param' \gets (\param,\sigma)$.
\item Run $b' \getsr \A(\param')$. Suppose $\A$ makes an oracle
query.
\begin{itemize}
\item If $\A$ makes a \texttt{CreateTA}, \texttt{SubmitTA},
\texttt{CoalitionBroadcast}, \texttt{CoalitionUpdate},
\texttt{CoalitionExtract} or \texttt{Corrupt} query, then $\B^{*}$
passes the query to its own oracle and returns the result.

\item If $\A$ makes an encryption oracle query on the equal-length
messages $(m_{0},m_{1})$, the pattern $P^{*}$, and the
coalition $\C^{*}$ then $\B^{*}$ computes $P'\phantom{}^{*}
\gets \texttt{Encode}(P^{*},\mathit{com}^{*})$, $d\getsr \{0,1\}$, $m'_{0} \gets
(m_{d},\mathit{dec}^{*})$, $m'_{1} \gets
(m_{d},0^{|\mathit{dec}^{*}|})$, and queries its encryption
oracle on $(m'_{0},m'_{1})$, the pattern $P'\phantom{}^{*}$,
and the coalition $\C^{*}$. It receives $C'\phantom{}^{*}$
from its oracle, and computes $\tau^{*} \gets
\textsc{Mac}_{K^{*}}(\C^{*}\|P\phantom{}^{*}\|C'\phantom{}^{*})$.
It returns
$(\mathit{com}^{*},\C^{*},P^{*},C'\phantom{}^{*},\tau^{*})$.

\item If $\A$ makes a decryption query on $(\mathit{com},\C,P,C',\tau)$
with $\mathit{com} \neq \mathit{com}^{*}$ for the identity
$\ID$ and the coalition $\C$, then $\B^{*}$ computes $P'
\gets \texttt{Encode}(P,\mathit{com})$, replaces the wildcards in
$P'$ with $1^{k}$ to form the identity $\ID'$ and requests
the decryption key $d_{\ID'}$ for $\ID'$. Since $\A$ can
only make decryption queries for coalitions for which the
adjustment parameter $v$ is known, $\B^{*}$ forms the
decryption key $c' \getsr
\texttt{CoalitionExtract}(d_{\ID'},v)$. $\B^{*}$ can use this
key to decrypt $C'$, and decrypt the rest of the ciphertext
as normal.

\item If $\A$ makes a decryption query on
$(\mathit{com}^{*},\C,P,C',\tau)$, then $\B^{*}$ checks
whether $\textsc{Mac}_{K^{*}}(\C\|P\|C') = \tau$. If so,
$\B^{*}$ outputs 1 and terminates. Otherwise, $\B$ returns
$\perp$ to $\A$.
\end{itemize}
\item $\B^{*}$ outputs 0
\end{enumerate}
This attacker is legal since it only queries the decryption oracle
on identities with $\mathit{com} \neq \mathit{com}^{*}$. If $b=0$
then $\B^{*}$ outputs 1 whenever $E_{2}$ occurs. If $b=1$ then
$\B^{*}$ outputs 1 whenever $E_{3}$ occurs. Hence,
$|Pr[E_{3}]-\Pr[E_{2}]|
\leq \adv{\B^{*}}{CPA}(k)$.

There exists an attacker $\B'$ such that $|\Pr[W_{3} |\neg E_{3}] -
\Pr[W_{2} |\neg E_{2}]| \leq \adv{\B'}{CPA}(k)$. The attacker
$\B'(\param)$ is defined as follows:
\begin{enumerate}
\item $\sigma \getsr \G(1^{k})$ and
$(K^{*},\mathit{com}^{*},\mathit{dec}^{*}) \getsr
\S(1^{k},\sigma)$.
\item $\param' \gets (\param,\sigma)$.
\item Run $d' \getsr \A(\param')$. Suppose $\A$ makes an oracle
query.
\begin{itemize}
\item If $\A$ makes a \texttt{CreateTA}, \texttt{SubmitTA},
\texttt{CoalitionBroadcast}, \texttt{CoalitionUpdate},
\texttt{CoalitionExtract} or \texttt{Corrupt} query, then $\B$
passes the query to its own oracle and returns the result.

\item If $\A$ makes an encryption oracle query on the equal-length
messages $(m_{0},m_{1})$ and the pattern $P^{*}$, then $\B$
computes $P'\phantom{}^{*} \gets
\texttt{Encode}(P^{*},\mathit{com}^{*})$, $d\getsr \{0,1\}$, $m'_{0} \gets
(m_{d},\mathit{dec}^{*})$, $m'_{1} \gets
(m_{d},0^{|\mathit{dec}^{*}|})$, and queries its encryption
oracle on $(m'_{0},m'_{1})$ and the pattern $P'$. It
receives $C'\phantom{}^{*}$ from its oracle, and computes
$\tau^{*}
\gets \textsc{Mac}_{K^{*}}(\C^{*}\|P'\phantom{}^{*}\|C'\phantom{}^{*})$. It
returns
$(\mathit{com}^{*},\C^{*},P^{*},C'\phantom{}^{*},\tau^{*})$.

\item If $\A$ makes a decryption query on $(\mathit{com},\C,P,C',\tau)$
with $\mathit{com} \neq \mathit{com}^{*}$ for the identity
$\ID$ and the coalition $\C$, then $\B$ computes $P' \gets
\texttt{Encode}(P,\mathit{com})$, replaces the wildcards in
$P'$ with $1^{k}$ to form the identity $\ID'$ and requests
the decryption key $d_{\ID'}$ for $\ID'$. Since $\A$ can
only make decryption queries for coalitions for which the
adjustment parameter $v$ is known, $\B$ forms the decryption
key $c'
\getsr \texttt{CoalitionExtract}(d_{\ID'},v)$. $\B$ can use this key to decrypt $C'$, and decrypt the
rest of the ciphertext as normal.

\item If $\A$ makes a decryption query on
$(\mathit{com}^{*},\C,P,C',\tau)$, then $\B$ returns
$\perp$.
\end{itemize}
\item If $d=d'$ then $\B'$ returns 1, else it returns 0.
\end{enumerate}
This is a legal attacker and $|\Pr[W_{3} |\neg E_{3}] -
\Pr[W_{2} |\neg E_{2}]| \leq \adv{\B'}{CPA}(k)$. A simple
probability argument can be used to show that:
\begin{displaymath}
|\Pr[W_{3}]-\Pr[W_{2}]| \leq 2\cdot\adv{\B^{*}}{CPA}(k) +
\adv{\B'}{CPA}(k) + \Pr[E_{3}]\, .
\end{displaymath}

Next, let Game 4 be identical to Game 3 except that the key $K^{*}$
used in the encryption algorithm (and to determine if ciphertexts
should be rejected) is randomly chosen from $\{0,1\}^{k}$. There
exists an attacker $\B^{\dagger}$ against the hiding property of the
encapsulation algorithm such that $|\Pr[W_{4}]-\Pr[W_{3}]| \leq
2\cdot\adv{\B^{\dagger}}{hide}(k)$. Let $E_{4}$ be the event that
$\A$ submits a ciphertext $(\mathit{com}^{*},\C,P,C',\tau)$ to the
decryption oracle with $\textsc{Mac}_{K^{*}}(\C\|P\|C')=\tau$ in
Game 4. Again, we have $|\Pr[E_{4}]-\Pr[E_{3}]|\leq 2\cdot
\adv{\B^{\dagger}}{hide}(k)$.

Finally, let Game 5 be identical to Game 4 except that (a) the
attacker loses if it queries the decryption oracle on a ciphertext
$(\mathit{com}^{*},\C,P,C',\tau)$ before it queries the encryption
oracle, and (b) the attacker returns $\perp$ whenever the attacker
queries the decryption oracle on a ciphertext
$(\mathit{com}^{*},\C,P,C',\tau)$ after it queries the encryption
oracle. There exists an algorithm $\B''$ against the MAC algorithm
such that $|\Pr[W_{5}]-\Pr[W_{4}]| \leq q_{D}\adv{\B''}{MAC}(k) +
\gamma(k)$ where $\gamma(k)$ is the maximum probability that a randomly
generated $\mathit{com}^{*}$ is any fixed binary value. As a
byproduct, we also obtain $\Pr[E_{4}]\leq q_{D}\adv{\B''}{MAC}(k)$.

We can show a direct reduction from Game 5 to the underlying IND-CPA
security of $\Pi$. There exists an algorithm $\B^{\sharp}$ such that
$2\cdot|\Pr[W_{5}]-1/2| = \adv{\B^{\sharp}}{CPA}(k)$. This algorithm
simply translates decryption oracle queries made by $\A$ against the
MTA-WIBE scheme into translates decryption oracles made by
$\B^{\sharp}$ against the tag-based encryption scheme (for
ciphertexts with $\mathit{com}\neq \mathit{com}^{*}$) or returns
$\perp$ (for ciphertexts with $\mathit{com}=\mathit{com}^{*}$). All
decryption oracle queries made by $\B^{\sharp}$ are legal as the
weak selective-tag IND-CPA security model allows for decryption
oracle queries for tags $\mathit{com}\neq \mathit{com}^{*}$. This
concludes the proof.
\qed

\end{document}
