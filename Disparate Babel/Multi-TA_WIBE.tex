% http://dev.baywifi.com/latex
\documentclass[10pt]{article}
\usepackage[utf8]{inputenc}
\usepackage[LY1]{fontenc}
\usepackage{amssymb, amsmath, fullpage, boxedminipage, listings}
\usepackage{baskerville}

\linespread{1.5}
\setlength\parskip{10pt}
\setlength\parindent{0 pt}

% This is now the recommended way for checking for PDFLaTeX:
\usepackage{ifpdf}
\ifpdf
\usepackage[pdftex]{graphicx}
\else
\usepackage{graphicx}
\fi

\title{\textbf{Wildcarding in a Multi-TA HIBE Setting}}

\author{Kent D. Boklan\footnote{Queens College}, Christopher Seaman\footnote{CUNY Graduate Center}}
\date{June, 2008}


\begin{document}
\maketitle

\begin{center}{\textrm{WORKING DRAFT}}\end{center}
\section*{Introduction \& Motivation}
Identity-based encryption is most often considered in the context of one-to-one communication within a single Trusted Authority (TA); every encrypted message is sent between two individual entities.  In this paper we consider a coalition of TA's desiring secure one-to-many communication, with each TA retaining security of it's secret keys.  This secrecy requirement is natural in the setting of dynamic coalition forming and dissolution.

One-to-many secure communication is a powerful cryptographic tool.  Take a MANETs setting for example, where transmission is much more expensive than computation, one-to-many communication allows for a single transmitted message to be read by any number of valid recipients within range.  In this paper, multiple recipients may decrypt the same message through the use of one or more "wildcards". A wildcard is a special character that may be used in an address in lieu of specifying a particular aspect of an identity, allowing anyone matching the non-wildcard portion to read the message.

In the single-TA case, the method of employing wildcards into hierarchical identity-based encryption structures (Abdallah et al., 2006) allows an individual at any level within one TA to send messages to entire levels within that TA.  The hierarchy of a TA could be as simple as an email address, every entity under the TA (say "school.edu") has a name. Considered as an IBE setting, one might desire to send a single message to an address of the form "name@school.edu", but a message pertaining to everyone at the school might be better sent to “*@school.edu” in a wildcard IBE setting. 

In the multi-TA case, we could consider schools as separate Trusted Authorities.  These TA's could agree upon a protocol such as the one described in this paper to allow secure one-to-many hierarchical IBE.  As such, a message addressed to the leadership of universities could be sent to "provost@*.edu" or a message for computer system administrators might be addresses “sysadmin@*.*”.  The wildcard method of multicast communication is limited by the structure of the hierarchy: it cannot distinguish between entities within a single hierarchical level.  For example, a single message to "*@school.edu" could be read by "alice@school.edu", "bob@school", and "eve@school.edu"; however, it would not be possible to send a single message to Alice and Bob without also allowing Eve to read it. Likewise, a message may be addressed to a single specified TA or to all TA's by wildcard; it is impossible to select multiple TA's to receive a message without making the message readable across all TA's.

In addition to one-to-many communication, this paper assumes that the coalition of TA's may change over time and allows those changes without compromising the security of communication or any TA's secret key.  TA's may be removed or added to the coalition with minimal configuration.  Upon a change in coalition make-up the participating TA's exchange public information, and based on non-public secret information they are able to communicate.  The private keys of subordinate entities of each TA must be updated, but each TA can accomplish this through a single broadcast message exclusively readable to members of that TA.  This flexible reconfiguration ability similarly allows a fixed set of TA's in coalition to schedule secure reconfigurations with minimal communication long before any secret key has a chance to become stale or compromised.

An interesting consequence and possible drawback of using a hierarchical IBE is that messages sent to a subordinate entity may be decrypted by direct ancestors of that entity.  Explicitly, a message to "bob@school.edu" could be read by the entity "school.edu", but not by "eve@school.edu" or by an entity at "university.edu".  The ability to decrypt is possible because an entity's ancestors are able to generate new secret keys for subordinates, so "school.edu" can make keys for "alice@school.edu", "bob@school.edu", ad infinitum.  With the use of a centralized key distributor it may be possible to avoid this vulnerability by choosing keys in such a way as to isolate the levels of hierarchy from each other.

\section*{The Scheme}

We extend the Boneh-Boyen hierarchical IBE model as adapted to include wildcards by Abdallah, et al.  We assume that a group of $n$ TA's, $($TA$_1$, TA$_2$, $\dots$, TA$_n)$, wish to establish a coalition using a wildcard HIBE system.  We further require that the TA's create a master secret such that no group of $(n-1)$ TA's may recover the secret. We assume that an individual node at the $k$th level of a TA's hierarchy has an identity consisting of a $k$-tuple of bit-string identifiers that we shall describe. We fix $k \le L$.  Each TA$_i$ is considered to be on the first level of it's own hierarchy.

For example, TA$_i$ may have identity $($"TA Name"$)$ while a subordinate on the second tier of that TA may have identity $($"TA$_i$ Name"$,$ "2nd Tier Name$_i$"$)$. We do not assume that every TA has the same depth, only that each has depth less than or equal to $L$.  When referring to a particular identifier $(ID_{i,1}, ID_{i,2}, \dots, ID_{i,k-1}, ID_{i,k}) = ($"TA$_i$ Name"$,$ "2nd Tier Name"$,\dots,$ "Direct Parent's Name" $,$ "$k$th Tier Name"$)$ we use the notation $ID_{i,j}$ to refer to the $j$th identifier in the $k$-tuple of an entity under TA$_i$.  We treat bit-string identifiers as names and integers interchangeably in the encryption process that we describe in this section.

Our setup requires groups $\mathbb{G}^+$, $\mathbb{Z}_q,$ and $\mathbb{G}_T$, a bilinear map $e: \mathbb{G}^+\times \mathbb{G}^+ \rightarrow \mathbb{G}_T$, a message space $M$, and a hash $H: \mathbb{G}_T \rightarrow M$.  The security of this scheme is based on the difficulty of Bilinear Decision Diffie-Hellman problem for the map $e$, and also the discrete logarithm problem in $\mathbb{G}^+$. In our application we will assume $\mathbb{G}^+$ is a finite subgroup of points $E[m]$ on a supersingular elliptic curve $E$, that $\mathbb{G}_T$ is the finite field that is the image of $e(E[m], E[m])$, and that the message space $M$ is $\{0, 1\}^t$ for some fixed $t$.  We assume that the TA's have agreed upon the elliptic curve over a fixed finite field and a point $P$ of large prime order $q$.

To initialize our multi-TA wildcard setup, we assume that the TA's have an agreed upon order. The TA's in some manner independent of the security of the system (perhaps sequentially) choose elements of the $2L + 2$-tuple:
$$\{g_1, g_2, u_{1,0}, u_{2,0}, \dots, u_{L, 0},u_{1,1}, u_{2,1}, \dots, u_{L,1}\}$$
where each component is a `random' multiple of $P$. This information is made public. We require that discrete log problems associated with $g_1$ and $g_2$ are difficult.

In the Boneh-Boyen model, each first tier entity would get a key based upon a master secret of the form $\alpha\cdot g_2$.In our system, we do not have a single root authority and do not allow any TA or $(n-1)$ TA's to know this master secret. We require that each TA$_i$ choose $\alpha_i, r_{i,j} \in \mathbb{Z}_q$ for $i,j \in \{1, \dots, n\}$. Denoting each TA$_j$'s identity as $ID_{j,1}$, TA$_i$ makes public the points:
\begin{align}
\alpha_i \cdot g_1\\
\alpha_i \cdot g_2 + r_{i,j} \cdot (u_{1,0} + ID_{j,1} \cdot u_{1,1})\text{ for all } i \not= j\\
r_{i,j}\cdot g_1 \text{ for all } i \not= j
\end{align}

Each TA publishes one (1), $n-1$ (2)'s, and $n-1$ (3)'s.  This makes for $n$ (1)'s, $n^2 - n$ (2)'s, and $n^2 - n$(3)'s, or a total of $2n^2 - n$ public points based on secret information.  As such, increasing the size of the coalition increases the number of public points each TA publishes, which in turn reduces the complexity of the discrete logarithm problem.

Each TA$_j$ derives their individual secret master key using their secret value $r_{j,j}$by adding elements of (2) and (3) as follows (note: summations run over the $i$ index as $j$ is fixed):
\begin{align*}
d_{TA_j} = (a_0, a_1) = (((\Sigma \alpha_i)\cdot g_2 + (\Sigma r_{i,j}) \cdot (u_{1,0} + ID_{j,1} \cdot u_{1,1})),(\Sigma r_{i,j}) \cdot g_1).
\end{align*}
The multi-TA public key is then $(\Sigma \alpha_i \cdot g_1)$ formed by summing the public formula (1) together.

(Similar to the Boneh-Boyen architecture, each TA’s secret key $d_{TA_j}$ (as above) is related to a master secret$(\Sigma \alpha_i)\cdot g_2$ although no TA has knowledge of $(\Sigma \alpha_i)\cdot g_2$ or $\Sigma \alpha_i$ directly.

Under each TA$_j$, private keys for subordinate entities are generated from $d_{TA_j}$. For a subordinate on level $k$,the private key is given recursively from its immediate ancestor's private key $d_{\text{ancestor}} = (a_0, a_1,\dots, a_{k-1})$. Using a randomly generated $r_k \in \mathbb{Z}_q$, the $k$th level subordinate's private key$d_{\text{subordinate}} = (a_0 + r_k\cdot (u_{k,0} + ID_{j,k}\cdot u_{k,1}), a_1, \dots, a_{k-1}, r_k \cdot g_1)$.

Encryption and decryption work as in the standard wildcard hierarchical scheme. To send a message $m$ to all level-$k$identities matching a pattern ${\bf P} = (P_1, \dots, P_k)$ where each $P_i$ is either an identifier or a wildcard\footnote{For all messages sent to entities on a certain level within a fixed TA, any direct ancestor of an addressee may decrypt the message due to the derivative nature of the key generation process. Given a TA-wide central authority for key generation, it is possible to create private keys such that each subordinate entity's encrypted messages could be kept secret from their ancestor entities.}, we say $i \in W({\bf P})$ if $P_i$ is a wildcard. (Fixed identities will be denoted $i \not\in W({\bf P})$.)  The sender chooses a ‘random’ element $t \in \mathbb{Z}_q$ and outputs the ciphertext $C = ({\bf P}, C_1, C_{2,i}, C_3, C_{4,i,0}, C_{4,i,1})$ as
$i$ ranges as follows:
\begin{align*}
C_1 = t \cdot g_1\\
C_{2,i} = t \cdot (u_{i,0} + (P_i) \cdot u_{i,1})\text{ for }i \not\in W({\bf P})\\
C_3 = m \otimes H(e((\Sigma \alpha_i) \cdot g_1, g_2)^t)\\
C_{4,i,j} = t \cdot u_{i,j}\text{ for }i \in W({\bf P})\text{ and }j \in \{0,1\}
\end{align*}
If the recipient list for the message is composed of exactly $n$ wildcards on this $k^{\textrm{th}}$ level, the cipher vector
will be composed of $k+n+2$ parts.

Decryption works by first compensating for the use of wildcards and then processing the message using the recipient's secret key. We note that anyone knowing $(\Sigma \alpha_i \cdot g_2)$ may decrypt the message by calculating $C_3 \otimes H (e(C_1, \Sigma \alpha_i \cdot g_2))$. No TA or subordinate knows this value. An intended recipient under TA$_j$ has an $ID = (ID_{j,1}, ID_{j,2}, \dots, ID_{j,k})$ that matches the pattern ${\bf P} = (P_1,
\dots, P_k)$. This recipient may decrypt the message using their secret key $d_{ID} = (a_0, \dots, a_k)$ by calculating a new $C_2^{\prime}$
element and the processing the message:
\begin{align*}
C_{2,i}^{\prime} = C_{2,i}\text{ for }i \not\in W({\bf P})\\
\\
C_{2,i}^{\prime} = C_{4,i,0} + ID_{j,i} \cdot C_{4,i,1}\text{ for }i \in W({\bf P})\\
\\
m = C_3 \otimes H\left(\frac{e(C_1, a_0)}{\Pi_{i=1}^k e(a_i, C_{2,i}^{\prime})}\right)
\end{align*}
\section*{Proofs of Security}
IND-sWID-CPA secure:  identical to the proof in Abdalla, et al. and we omit the details here.

Addition of a signature scheme gives IND-sWID-CCA (as in Abdalla)

%Need to quantify security loss for having multiple TA's and levels, etc.

\section*{Reconfiguration}

Should the members of the coalition of TA's change, an interesting aspect of this system is that it is quickly reconfigurable.  Reconfiguration of the system in play relies on the hierarchical organization of secret keys, the ability to broadcast messages to all members of a TA.  The security of reconfiguration relies on only the security assumptions made previously: difficulty of the BDDH under e and the discrete logarithm problem in $\mathbb{G}^+$.
 
Once a new coalition is determined, the members choose $\beta_i$ and $s_{i,j}$ to replace $\alpha_i$ and $r_{i,j}$ and publish values of $(\beta_i \cdot g_1)$ for all i and $(\beta_i \cdot g_2 + s_{i,j} \cdot (u_{1,0} + ID_{j,1}\cdot u_{1,1}))$for all $i \neq j$.  As before, each TA$_j$ may then calculate their private key based on the withheld $i = j$ value, and replace the previous secret key $(a_0, a_1) = (((\Sigma \alpha_i)\cdot g_2 + \Sigma r_{i,j} \cdot (u_{1,0} + ID_{j,1}\cdot u_{1,1})), \Sigma r_{i,j} \cdot g_1)$ with the new secret key $d_{TA_j} = (b_0, b_1) = (\Sigma \beta_i \cdot g_2
 + \Sigma s_{i,j} \cdot (u_{1,0} + ID_{j,1}\cdot u_{1,1}), \Sigma s_{i,j} \cdot g_1)$ (summation runs on the $i$ index). The public key $\Sigma \beta_i \cdot g_1$ is also calculable from the public information during this setup and should be assumed available to all TA's and subordinates.   To disseminate the new private information, the TA's could calculate an adjustment term:
\begin{align*}
(b_0 - a_0, b_1 - a_1) = ((\Sigma \beta_i - \Sigma \alpha_i) \cdot g_2 + (\Sigma s_{i,j} - \Sigma r_{i,j})\cdot (u_{1,0} + ID_{j,1} \cdot u_{1,1}), (\Sigma s_{i,j} - \Sigma r_{i,j})\cdot g_1)
\end{align*}

The savings in reconfiguration costs come from TA’s needing to do a round of communication at the highest level and then allowing a broadcast of information to subordinates rather than being required to send a separate message to each subordinate using their unique private keys.

\section*{Properties of the Scheme}

Our setup enables a group of TA's to use identity based encryption in a mixed trust situation. The "super secret" $\Sigma \alpha_i \cdot g_2$ is inseparable from the secret values $r,_{i,j}$ and $\alpha_i$ that the TA's do not share. In this way, no TA has enough information to decrypt messages destined for another.  It should be noted that we inherit from a hierarchical cryptosystem the property that every subordinate's key is derived from its superior. From this  it follows that every superior may read messages sent to any subordinates in the hierarchy,not just immediate subordinates.  If there were an entity above the TA's, corresponding to the "super secret",it would be able to read all messages sent in the system; for this reason we force this value to be difficult to recover by any TA or group of TA's.

Collaborative secret creation requires revealing many pieces of information but results in a highly collusion-resistant secret. The generation process requires each TA to reveal $n$ pieces of information which are each related to the secret $\alpha_i$ that the TA must not reveal. From the security of each TA’s $\alpha_i$, any two TA’s colluding have no advantage over a single TA in recovering the unknown $\alpha_i$. Because no one but the target TA has this information, bringing more TA’s into an attacking coalition would bring no advantage either. This can be seen by imagining the worst case scenario, $(n-1)$ TA’s attacking the secret of the lone target TA. In this case, the attacker has knowledge of all $\alpha_i$’s and $r_{i,j}$’s except for $\alpha_k$ and $r_{k,k}$ for target TA$_k$. The attackers have knowledge of $((\Sigma \alpha_i)\cdot g_2+ r_{i,k} \cdot (u_{1,0} + ID_{k,1}\cdot u_{1,1}))$ for all $i \neq k$. From their secret and public pieces of information  the attackers would need to calculate $(\alpha_k \cdot g_2 + r_{k,k} \cdot ( u_{1,0} +ID_{k,1} \cdot u_{1,1}))$, but they cannot because $r_{k,k}$ is secret and $alpha_k$ is occluded by the discrete logarithm problem.
Similarly, the use of independent secrets in the setup ensures that no TA can read another’s messages without knowledge of either $\Sigma \alpha_i)\cdot g_2$ or the target TA’s individual $\alpha_i$. Solving either of these problems is equivalent to solving the Bilinear Diffie-Hellman problem. In the first case the attacker must be able to separate the two terms of the published $(\alpha_i \cdot g_2 + r_{i,j} \cdot (u_{1,0} + (ID_{j,1})\cdot u_{1,1}))$, which relies on knowledge of $r_{i,j}$’s. In the second case the attacker must recover $\alpha_i$, from solving a discrete logarithm on public information such as $\alpha_i \cdot g_1$.

The parameters used in encryption and decryption in this scheme does not depend on the TA membership of the sender or recipient(s) of a message (except in the sense of recipient identity). There are no TA-specific pieces of public information necessary as is sometimes the case in identity-based encryption with multiple TA’s. As such, ciphertext messages may be unencumbered by information about the sender. The system is truly “identity-based” because a ciphertext depends only on the identity of the recipient(s). 
One interesting question in this scenario is the security of sending a message across all TA’s. Is it possible for an attacker to masquerade as an additional TA by using the public information available?
It is interesting to note that this scheme allows for broadcast of information to many (or all) identities within TA’s. The incorporation of wildcards into the scheme comes at the (low) cost of having more public parameters for the crypto-system.


\end{document}