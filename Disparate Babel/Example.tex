%
%  untitled
%
%  Created by Christopher Seaman on 2008-06-03.
%  Copyright (c) 2008 __MyCompanyName__. All rights reserved.
%
\documentclass[10pt]{article}
\usepackage[utf8]{inputenc}
\usepackage[T1]{fontenc}
\usepackage{amssymb, amsmath, fullpage, boxedminipage, listings}
\usepackage{texnansi, gtamacbaskerville}
\linespread{1.5}
\setlength\parskip{10pt}
\setlength\parindent{0 pt}

% This is now the recommended way for checking for PDFLaTeX:
\usepackage{ifpdf}
\ifpdf
\usepackage[pdftex]{graphicx}
\else
\usepackage{graphicx}
\fi
\title{Multi-TA Wildcard HIBE by Example}
\author{  }

\date{ June 2008 }

\begin{document}

\ifpdf
\DeclareGraphicsExtensions{.pdf, .jpg, .tif}
\else
\DeclareGraphicsExtensions{.eps, .jpg}
\fi

\maketitle


\section*{Setup}

In a setting of three TA's creating a multi-TA wildcard HIBE, name the TA's as "TA$_1$", "TA$_2$", and "TA$_3$" respectively.  The TA's agree upon groups $\mathbb{G}^+$, $\mathbb{Z}_q,$ and $\mathbb{G}_T$, and a bilinear map $e: \mathbb{G}^+ \times \mathbb{G}^+ \rightarrow \mathbb{G}_T$.  Concretely, $\mathbb{G}^+$ is a finite subgroup of points $E[m]$ on a supersingular elliptic curve $E$ generated by a public point $P$ of large prime order $q$, that $\mathbb{G}_T$ is the finite field that is the image of $e(E[m], E[m])$.  The TA's generate the public $2L + 2$-tuple of 'random' multiples of the point $P$:
$$\{g_1, g_2, u_{1,0}, u_{2,0}, \dots, u_{L, 0},u_{1,1}, u_{2,1}, \dots, u_{L,1}\}$$

\section*{Generating a Secret}
We require that each TA$_i$ choose $\alpha_i, r_{i,j} \in \mathbb{Z}_q$ for $i,j \in \{1, 2, 3\}$. Denoting each 
TA$_j$'s identity as $ID_{j,1}$, TA$_i$ makes public the points:
\begin{eqnarray}
	\alpha_i \cdot g_1\\
	\alpha_i \cdot g_2 + r_{i,j} \cdot (u_{1,0} + ID_{j,1} \cdot u_{1,1})\text{ for all } i \not= j\\
	r_{i,j}\cdot g_1 \text{ for all } i \not= j
\end{eqnarray}

Each TA publishes one (1), two (2)'s, and two (3)'s.  This makes for three (1)'s, six (2)'s, and six (3)'s, for a total of fifteen public points based on secret information.  The multi-TA public key is then $(\alpha_1 \cdot g_1 + \alpha_2 \cdot g_1 + \alpha_3 \cdot g_1) = ( (\alpha_1 + \alpha_2 + \alpha_3) \cdot g_1)$ formed by summing the public formula (1) together.  Each TA generates a secret key by adding elements of (2) and (3) as follows (note: summations run over the $i$ index as $j$ is fixed):
\begin{align*}
d_{TA_j} = (a_0, a_1) = (((\Sigma \alpha_i)\cdot g_2 + (\Sigma r_{i,j}) \cdot (u_{1,0} + ID_{j,1} \cdot u_{1,1})), 
(\Sigma r_{i,j}) \cdot g_1).
\end{align*}
For example, calculating the key for TA$_2$:
$$d_{TA_2} = (((\alpha_1 + \alpha_2 + \alpha_3) \cdot g_2 + (r_{1,2} + r_{2,2} + r_{3,2}) \cdot (u_{1,0} + ID_{2,1} \cdot u_{1,1})), (r_{1,2} + r_{2,2} + r_{3,2}) \cdot g_1)$$

\section*{Distributing Keys to Subordinates}
The secret keys for all subordinates of a TA are derived from that TA's secret key.  Let Alice be a direct subordinate to TA$_2$ with her full identity as $(ID_{2,1}, ID_{2,2})$.  For a 'random' $r \in \mathbb{Z}_q$ 

\section*{Sending a Message}

\section*{Reconfiguration}


\end{document}














