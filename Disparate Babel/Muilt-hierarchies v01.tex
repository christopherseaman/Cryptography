\documentclass{llncs}

\usepackage{a4wide}
\usepackage{amsmath,amssymb}
\usepackage[LY1]{fontenc}
\usepackage{baskerville}

\newcommand{\A}{\mathcal{A}}

\newcommand{\ID}{\mathit{ID}}
\newcommand{\TA}{\mathit{TA}}

\newcommand{\mpk}{\mathit{mpk}}
\newcommand{\msk}{\mathit{msk}}

\newcommand{\getsr}{\stackrel{{\scriptscriptstyle\$}}{\gets}}
\newcommand{\adv}[2]{\mathit{Adv}_{#1}^{\tiny \texttt{#2}}}

\title{Multiple Hierarchy Wildcard Encryption}
\author{All Who Want To Play}
%\institute{...}

\begin{document}
\maketitle

\begin{abstract}
expressionism is an art movement typically characterised by its
non-realistic representation of non-tangible nouns (such as emotions
or situations). It centred in the New York in the post-surrealistic
decades after the second World War, and, in particular, around Peggy
Guggenheim's gallery ``Art of this Century''.
\end{abstract}

\section{Syntax}

A multi-hierarchy WIBE consists of the following PPT
algorithms/protocols:
\begin{itemize}
\item $\texttt{Setup}(1^{k},\TA)$: This algorithm is run
once by a TA and outputs a master public key and master private key
for that TA $(\mpk_{\TA},\msk_{\TA})\getsr
\texttt{Setup}(1^{k},\TA)$. \medskip

\emph{What's to prevent an attacker setting up his own TA under the
name of a real coalition member and then hijacking the update
coalition protocol? Are we going to assume trusted distribution of
master public keys?}\medskip

\item
$\texttt{SetupCoalitionBroadcast}(\TA,\msk_{\TA},(\TA_{1},\mpk_{\TA_{1}}),\ldots,
(\TA_{n},\mpk_{\TA_{n}}))$: This algorithm creates a coalition
between a set of TAs $\mathcal{C} = (\TA_{1},\ldots,\TA_{n})$. This
algorithm outputs a list of messages $w_{i}$ to be sent to the TA
$\TA_{i}$
$((\TA_{1},w_{\TA_{1}}),\ldots,(\TA_{n},w_{\TA_{n}}))$.\medskip

\item $\texttt{SetupCoalitionKeys}(\TA,\msk_{\TA},(\TA_{1},\mpk_{\TA_{1}},w_{1}),\ldots,
(\TA_{n},\mpk_{\TA_{n}},w_{n}))$: The algorithm completes the setup
of the coalition. After every member $\TA_{i}$ of the coalition has
provided a message $w_{i}$ for $\TA$. It outputs a message $u_{\TA}$
to be broadcast to every member of its hierarchy.
\end{itemize}
The system should be able to dynamically update the coalition. We
may wish to change a coalition $\mathcal{C}$ into a coalition
$\mathcal{C}'$. We assume that members $\mathcal{C} \cap
\mathcal{C}'$ execute the $\texttt{UpdateCoalition}$ algorithms,
while new members $\mathcal{C}\setminus \mathcal{C}'$ execute the
$\texttt{JoinCoalition}$ algorithms. Excluded members
$\mathcal{C}'\setminus \mathcal{C}$ are simply informed that they
are no longer members of the coalition.
\begin{itemize}
%\item
%$\texttt{AddCoalitionBroadcast}(\TA,\msk_{\TA},(\TA'_{1},\mpk_{\TA'_{1}}),\ldots,(\TA'_{n'},\mpk_{\TA_{n'}}))$:
%This algorithm adds authorities $(\TA'_{1},\ldots,\TA'_{n'})$ to an
%existing coalition. Each existing coalition member $\TA$ runs this
%algorithm and sends the result messages $(w'_{1},\ldots,w'_{n'})$ to
%the appropriate new TAs.\medskip
%
%\item
%$\texttt{AddCoalitionKeys}(\TA,\msk_{\TA},(\TA'_{1},\mpk_{\TA'_{1}},w'_{1}),\ldots,(\TA'_{n'},\mpk_{\TA_{n'}},w'_{n}))$:
%This algorithm outputs a message $u_{\TA}$ to be broadcast to its
%own hierarchy in order to re-key the coalition.\medskip

\item $\texttt{UpdateCoalitionBroadcast}(\TA,\msk_{\TA},(\TA_{1},\mpk_{\TA_{1}}),\ldots,
(\TA_{n},\mpk_{\TA_{n}}))$: This algorithm updates an existing
coalition $\mathcal{C}$ contain $\TA$ to become a new coalition
$\mathcal{C}'=(\TA,\TA_{1},\ldots,\TA_{n})$. This algorithm outputs
a list of messages $w_{i}$ to be sent to the TA $\TA_{i}$. It should
be noted that some $w_{i}$ may be empty, particularly if $\TA_{i}\in
\mathcal{C}$.\medskip

\item $\texttt{Join CoalitionBroadcast}(\TA,\msk_{\TA},(\TA_{1},\mpk_{\TA_{1}}),\ldots,
(\TA_{n},\mpk_{\TA_{n}}))$: A new authority $\TA$ which is joining
an existing coalition to form a new coalition
$\mathcal{C}'=(\TA,\TA_{1},\ldots,\TA_{n})$ uses this algorithm to
produce a series of messages $w_{i}$ to be sent to
$\TA_{i}$.\medskip

\item $\texttt{UpdateCoalitionKeys}(\TA,\msk_{\TA},(\TA_{1},\mpk_{\TA_{1}},w_{1}),\ldots,
(\TA_{n},\mpk_{\TA_{n}},w_{n}))$: The algorithm completes the
updating of the coalition for existing members. After every member
$\TA_{i}$ of the coalition has provided a (non-empty) message
$w_{i}$ for $\TA$. It outputs a message $u_{\TA}$ to be broadcast to
every member of its hierarchy.\medskip

\item $\texttt{JoinCoalitionKeys}(\TA,\msk_{\TA},(\TA_{1},\mpk_{\TA_{1}},w_{1}),\ldots,
(\TA_{n},\mpk_{\TA_{n}},w_{n}))$: This algorithm completes the
joining of an existing coalition for new members. After every member
$\TA_{i}$ of the coalition has provided a (non-empty) message
$w_{i}$ for $\TA$, this algorithm outputs a message $u_{\TA}$ to be
broadcast to every member of its hierarchy.
\end{itemize}
We now describe the algorithms required by the individual users.
\begin{itemize}
\item $\texttt{Extract}(\vec{ID},\ID',d_{\vec{\ID}})$: This
algorithm outputs a decryption key $d_{\vec{\ID}\|\ID}$ for the
identity $\vec{\ID} \| \ID$. The basic level has $\ID=\TA$ and
$d_{\TA}=\msk_{\TA}$. \medskip

\item
$\texttt{ExtractCoalitionKey}((\TA_{1},\ldots,\TA_{n}),u_{\TA},d_{\ID})$:
This algorithm outputs a user key $c_{\ID}$ for the coalition
$\mathcal{C}=\{\TA,\TA_{1},\ldots,\TA_{n}\}$ by combining the
broadcast key $u_{\TA}$ and their decryption key $d_{\ID}$.\medskip

\item $\texttt{UpdateCoalitionKey}((\TA_{1},\ldots,\TA_{n}),u_{\TA},c_{\ID},d_{\ID})$:
This algorithm outputs an updated user key $c'_{\ID}$ for the
coalition $\mathcal{C}=\{\TA,\TA_{1},\ldots,\TA_{n}\}$ by combining
the broadcast key $u_{\TA}$ with the user's decryption key $d_{\ID}$
and existing coalition key $c_{\ID}$.\medskip

\item $\texttt{Encrypt}((\TA_{1},\mpk_{\TA_{1}}),\ldots,(\TA_{1},\mpk_{\TA_{1}}),\mathit{P},m)$:
This algorithm is used to encrypt a message $m$ to entities
satisfying the pattern $\mathit{P}$ under the coalition formed by
$(\TA_{1},\ldots,\TA_{n})$. It outputs a ciphertext $C$ or the
invalid symbol $\perp$.
\medskip

\item $\texttt{Decrypt}$: It does what you'd expect...
\end{itemize}

\section{Security Model}

The security model is parameterised by a bit $b$ involves a PPT
attacker $\A$ which is initially given the input $1^k$ and access to
the following oracles:
\begin{itemize}
\item $\texttt{CreateTA}(\TA)$: The oracle computes
$(\mpk_{\TA},\msk_{\TA})\getsr \texttt{Setup}(1^{k},\TA)$ for the TA
identity $\TA$ and returns $\mpk_{\TA}$. This oracle can only be
queried once for each identity $\TA$.\medskip

\item
$\texttt{SetupCoalitionBroadcast}(\TA,(\TA_{1},\ldots,\TA_{n}))$:
This oracle runs the $\texttt{SetupCoalitionBroadcast}$ algorithm on
the appropriate inputs and returns $(w_{1},\ldots,w_{n})$.\medskip

\item $\texttt{SetupCoalitionKeys}(\TA, (w_{1},\ldots,w_{n}))$: This
oracle can only be queried if $\TA$ has been queried to the
$\texttt{SetupCoalitionBroadcast}$ oracle with $n$ $\TA$'s in the
coalition. The oracles runs the $\texttt{SetupCoalitionKeys}$
algorithm assuming that message $w_{i}$ was sent by $\TA_{i}$. Note
that this does not imply that all the TAs believe that they're in
the same coalition.\medskip

\item $\texttt{UpdateCoalition}$ oracles are similar to the
above...\medskip

\item $\texttt{CorruptTA}(\TA)$: The oracle returns $\msk_{\TA}$ and
records that $\TA$ is corrupt.\medskip

\item $\texttt{CorruptUser}(\TA,\vec{\ID})$: This oracle returns
$d_{\vec{\ID}}$ for the identity $\vec{\ID}$ under the authority
$\TA$. Note that given $d_{\vec{\ID}}$ the attacker can compute any
coalition key $c_{\vec{\ID}}$. \medskip

\item $\texttt{UserDecrypt}(\TA,\vec{\ID},C^{*})$: This oracle
decrypts the ciphertext with the decryption key
$d_{\vec{\ID}}$.\medskip

\item $\texttt{CoalitionDecryption}(\TA,\vec{\ID},C^{*})$: This oracle
decrypts the ciphertext with the decryption key
$c_{\vec{\ID}}$.\medskip

\item
$\texttt{Test}(\TA_{1},\ldots,\TA_{n},\mathit{P},m_{0},m_{1})$: This
oracle takes as input two messages $(m_{0},m_{1})$ of equal length.
It encrypts the message $m_{b}$ for the coalition using
$(\mpk_{\TA_{1}},\ldots,\mpk_{\TA_{n}})$ under the pattern
$\mathit{P}$. This oracle may only be access once and outputs a
ciphertext $C^{*}$. We will let $\mathcal{C}^{*}$ denote the
challenge coalition $(\TA_{1},\ldots,\TA_{n})$.
\end{itemize}
The attacker terminates by outputting a bit $b'$. The attacker's
advantage is defined to be:
\begin{displaymath}
%\adv{\A}{IND}
\mathit{Adv}_{\A}^{\texttt{IND}}
(k) = |Pr[b'=1|b=1]-Pr[b'=1|b=0]| \, .
\end{displaymath}
The disallowed oracle queries: (1) a \texttt{CorruptTA} query for
any TA in the test coalition , (2) a \texttt{CorruptUser} query for
any user $\ID$ matching the pattern $\mathit{P}$ under an authority
$\TA$ in the test coalition if there has been a
\texttt{SetupCoalitionKeys} or \texttt{UpdateCoalitionKeys} query
for the test coalition, (3) a decrypt query for $C^{*}$ and any user
$\ID$ matching the pattern $\mathit{P}$ under an authority $\TA$ in
the test coalition if there has been a \texttt{SetupCoalitionKeys}
or \texttt{UpdateCoalitionKeys} query for the test coalition.

\bibliographystyle{plain}
\bibliography{../../Master}
\end{document}
